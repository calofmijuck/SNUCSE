%!TEX encoding = utf-8
\documentclass[12pt]{article}
\usepackage{amsmath}
\usepackage{amsfonts}
\usepackage{mathtools}
\usepackage{kotex}
\usepackage{geometry}
	\geometry{
		top = 30mm,
		left = 40mm,
		right = 40mm,
	}

\pagenumbering{gobble}
\renewcommand{\baselinestretch}{1.3}

\begin{document}
\begin{center}
\textbf{Quiz 1 (9월 15일)}\\
{[고급수학 및 연습 2 - 2017학년도 2학기]}\\
(제한시간: 20분, 만점: 20점)\\
\end{center}

* 답안지에 학번과 이름을 쓰시오. 답안 작성시 풀이과정을 명시하시오.\\

\begin{enumerate}
\item 연속함수
$$
	f(x, y) = \begin{cases}
    \dfrac{x^4+y^4}{x^2+y^2},       & \quad (x, y) \neq (0, 0)\\
    0,  & \quad (x, y) = (0, 0)
  \end{cases}
$$ 의 미분가능성을 조사하시오. (7점) 
~\\
\item 함수 $f(x, y, z)=x^y+y^z+z^x$ 의 $3$ - 등위면에 대하여, 점 $(1, 1, 1)$에서의 접평면의 방정식을 구하시오. (4점)
~\\
\item 함수 $$f(x, y)= \int_{xy}^{\sqrt{x^2+y^2}} \log{(1+t^2)} dt$$ 에 대하여 $\displaystyle \frac{\partial f}{\partial x}(1, 0)$ 과 $\displaystyle \frac{\partial f}{\partial y}(0, 1)$ 을 구하시오. (4점)
~\\\\
\item 모든 점 $P\in \mathbb{R}^3$ 에 대하여, 일급함수 $f$ 의 기울기 벡터 $\text{grad} f(P)$ 는 단위벡터이고, 원점을 중심으로 하는 $P$ 를 지나는 구면에 수직이다. 함수 $f$ 를 구하시오. (5점)

\end{enumerate}

\end{document}