%!TEX encoding = utf-8
\documentclass{article}
\usepackage{amsmath}
\usepackage{amsfonts}
\usepackage{mathtools}
\usepackage{kotex}
\usepackage{geometry}
	\geometry{
		top = 30mm,
		left = 40mm,
		right = 40mm,
	}

\pagenumbering{gobble}
\renewcommand{\baselinestretch}{1.3}

\begin{document}
\begin{center}
\textbf{Quiz 3}\\
{[고급수학 및 연습 1 (003) - 2017학년도 1학기]}\\
(제한시간: 20분, 만점: 20점)\\
\end{center}

* 답안지에 학번과 이름을 쓰시오. 답안 작성 시 풀이과정을 명시하시오.

\begin{enumerate}
\item $\mathbb{R}^3$ 위의 평면 $\pi: x+y+z=0$ 와 직선 $l(t) = t(1, 2, 0) \quad (t\in\mathbb{R})$이 주어져 있을 때 다음 물음에 답하시오.
	\begin{enumerate}
		\item[(a)] (3점) $\mathbb{R}^3$의 점 $X$를 평면 $\pi$에 정사영하는 사상 $L_1: \mathbb{R}^3 \rightarrow \mathbb{R}^3$과, $\mathbb{R}^3$의 점 $X$를 직선 $l$에 대해 대칭시키는 사상 $L_2: \mathbb{R}^3 \rightarrow \mathbb{R}^3$가 선형사상임을 보이시오. \\
		\item[(b)] (4점) $L_1, L_2$에 해당하는 행렬 $A_1, A_2$를 구하시오. \\
		\item[(c)] (3점) $\mathbb{R}^3$의 점 $X$를 평면 $\pi$에 정사영한 뒤, 그 점을 다시 직선 $l$에 대해 대칭시킨 점을 $Y$라 하자. 사상 $L_3: \mathbb{R}^3 \rightarrow \mathbb{R}^3$가 $L_3 X = Y$로 주어졌을 때, $L_3$에 해당하는 행렬을 $A_1, A_2$로 나타내시오.
	\end{enumerate}
~
\item 다음 물음에 답하여라.
	\begin{enumerate}
		\item[(a)] (5점) 자연수 $n$에 대해 $a_1, a_2, \cdots, a_n$이 서로 다른 실수일 때, $n$개의 벡터 \\$(1, a_1, a_1^2, \cdots, a_1^{n-1}), (1, a_2, a_2^2, \cdots, a_2^{n-1}), \cdots, (1, a_n, a_n^2, \cdots, a_n^{n-1})$이 일차독립임을 증명하시오.\\
		\item[(b)] (5점) 주어진 $n\times n$ 행렬 $A$와 영벡터가 아닌 $n$개의 $n$-벡터 $v_1, \cdots, v_n$이 서로 다른 실수 $a_1, a_2, \cdots, a_n$에 대해
		$$Av_i = a_iv_i, \qquad i = 1, 2, \cdots, n$$ 을 만족할 때, $v_1, \cdots, v_n$이 일차독립임을 증명하시오.
	\end{enumerate}


\end{enumerate}

\end{document}