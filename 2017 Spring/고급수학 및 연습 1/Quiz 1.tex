%!TEX encoding = utf-8
\documentclass{article}
\usepackage{amsmath}
\usepackage{mathtools}
\usepackage{kotex}
\usepackage{geometry}
	\geometry{
		top = 30mm,
		left = 40mm,
		right = 40mm,
	}

\pagenumbering{gobble}
\renewcommand{\baselinestretch}{1.3}

\begin{document}
\begin{center}
\textbf{Quiz 1}\\
제한시간: 20분, 만점: 20점\\
{[고급수학 및 연습 1 (003) - 2017학년도 1학기]}\\
\end{center}

* 답안지에 학번과 이름을 쓰시오. 답안 작성 시 풀이과정을 명시하시오.

\begin{enumerate}
\item 수열 $\left\{ a_n \right\} _{n \geq 1}$이 다음과 같이 주어져 있을 때, 급수 $\displaystyle\sum\limits_{n=1}^{\infty}a_n$의 수렴 혹은 발산 여부를 확인하시오.
\\
(a) (4점) $\displaystyle a_n=\frac{1}{3n^2-20}$.
\\\\
(b) (4점) $\displaystyle a_n=\left(\frac{1}{n} - 1\right) ^n$.
\\

\item (6점) 다음 급수가 수렴함을 증명하시오.
\begin{equation*}
\sum_{n=1}^{\infty}a_n, \quad\quad a_n=\left\{
                \begin{array}{rl}
                  \dfrac{2}{n}, &\, n\text{은 }  3\text{의 배수}\\ \\
 			-\dfrac{1}{n}, &\, n\text{은 } 3\text{의 배수가 아닌 자연수.}
                \end{array}
              \right.
\end{equation*}
\\
\item (a) (3점) $f(x)=\dfrac{1}{\sqrt{x}}$의 그래프를 이용하여 임의의 자연수 $n$에 대해
$$\frac{1}{\sqrt{2}}+\frac{1}{\sqrt{3}}+\cdots+\frac{1}{\sqrt{n}}<2\sqrt{n}-2<\frac{1}{\sqrt{1}}+\frac{1}{\sqrt{2}}+\cdots+\frac{1}{\sqrt{n-1}}$$
임을 확인하시오.\\\\
(b) (3점) 위의 결과를 사용하여 극한값
$$\delta := \lim_{n \rightarrow \infty} \left(\sum_{k=1}^n \frac{1}{\sqrt{k}}-2\sqrt{n}\right)$$
이 존재함을 보여라.
\end{enumerate}

\end{document}