%!TEX encoding = utf-8
\documentclass[12pt]{report}
\usepackage{kotex}
\usepackage{amsmath}
\usepackage{amsfonts}
\usepackage{amssymb}
\usepackage{mathtools}
\usepackage{geometry}
	\geometry{
		top = 30mm,
		left = 30mm,
		right = 30mm,
		bottom = 30mm
	}

\pagenumbering{gobble}
\renewcommand{\baselinestretch}{1.3}
\newcommand*{\trans}{^{\mathrm{\mathbf{t}}}}%
\newcommand*{\basisB}{\mathfrak{B}}%
\newcommand*{\basisC}{\mathfrak{C}}%
\newcommand*{\basisD}{\mathfrak{D}}%
\newcommand*{\matrixM}{\mathfrak{M}}%
\newcommand*{\im}{\text{im}\:}%
\newcommand\tab[1][5mm]{\hspace*{#1}}

\begin{document}
\begin{center}
\textbf{\Large선형대수학 과제 4}\\
\large 2017-18570 이성찬
\end{center}

\begin{itemize}
\item[\textbf{7.4.7}] (Cayley-Hamilton Theorem 첫 번째 증명) 교재에서와 마찬가지로 $$\phi_T(T) = (T-\lambda_1I)(T-\lambda_2I)\cdots(T-\lambda_nI)=0$$ 을 보이면 된다. 위 식에서 왼쪽의 $k (\leq n)$개 항을 곱해 얻은 행렬의 왼쪽 $k$개의 column 이 zero column 임을 보이면 충분하다. 수학적 귀납법을 활용한다.\\
	\tab(i) $k=1$ 일 때, $T-\lambda_1I$의 첫 번째 column 은 zero column 이다. 그리고 이 행렬은 upper-triangular 이다. \\
	\tab(ii) $k$ $(< n)$ 일 때, $A = (T-\lambda_1I)(T-\lambda_2I)\cdots(T-\lambda_kI)$ 의 왼쪽 $k$개의 column 이 zero column 이라고 가정하자. 연습문제 1.1.7 (가)에 의하면 두 upper-triangular matrix의 곱은 upper-triangular matrix 이기 때문에, $A = (a_{ij})$ 는 upper-triangular 이다. $B = T-\lambda_{k+1}I = (b_{ij})$ 로 두고 $AB = (c_{ij})$ 가 upper-triangular 이고 왼쪽 $k+1$개의 column 들이 모두 zero column 임을 보이면 된다. $j \leq k+1$ 일 때,	
	$$\begin{aligned}
		c_{ij} &= \sum_{x=1}^n a_{ix}b_{xj} = \sum_{x=1}^k a_{ix}b_{xj} + \sum_{x=k+1}^n a_{ix}b_{xj} \\ &=\sum_{x=1}^k 0\cdot b_{xj} + \sum_{x=k+1}^n a_{ix} \cdot 0 = 0
	\end{aligned}$$ ($a_{ix} = 0$ for $x\leq k$, 그리고 $b_{x,k+1} = 0$ for $k+1\leq x \leq n$) \\따라서 $k+1$ 일 때도 성립한다. 그러므로 $\phi_T(T)$의 왼쪽 $n$개의 column 들은 전부 zero column 이므로 $\phi_T(T)=0$.
	
\item[\textbf{8.1.8}] $m_T(t)$는 $p_i(t)$ 들을 최소한 한 번씩은 포함해야 한다. 따라서 각 기약다항식 $p_i(t)$ 마다 가질 수 있는 지수는 $1, \cdots, e_i$ 이므로 가능한 경우의 수는 $\prod_{i=1}^ke_i$.

\item[\textbf{8.2.12}] 
	\begin{itemize}
		\item[\textbf{(가)}] Let $f(t)=a_nt^n+\cdots+a_1t+a_0$. $T|_W$를 행렬로 생각하면 $$f(T|_W)=a_n(T|_W)^n+\cdots + a_1T|_W+a_0I$$ 이다. 이제 $f(T|_W)$ 와 $f(T)|_W$ 를 $w\in W$ 에서 evaluate 한 값이 같음을 보이면 된다. $W$가 $T$-invariant subspace 임을 이용하면 $(T|_W)^nw = T^nw$ 이므로,$$\begin{aligned} f(T|_W)w &= a_n(T|_W)^nw+\cdots + a_1T|_Ww+a_0w \\&= a_nT^nw + \cdots + a_1Tw + a_0w = (a_nT^n + \cdots + a_1T + a_0I)w \\ &=f(T)|_Ww\end{aligned}$$ 가 되어 성립한다.
		\item[\textbf{(나)}] $W$가 $T$-stable 인 것은 8.2.10 (나) 로부터 알 수 있다. $\forall w\in W$, $g(T)w=0$ 이어야 한다. $g(T)|_Ww = g(T)w = 0$ 이며, (가)에 의해 $g(T)|_W = g(T|_W)$ 이므로 $g(T|_W)w=0$. 따라서 $g(T|_W) = 0$ 이며, minimal polynomial $m_{T|_W}(t)$는 당연히 $g(t)$를 나눠야 한다.
	\end{itemize}
	
\item[\textbf{8.3.8}] 
	\begin{itemize}
		\item[\textbf{(가)}] Characteristic polynomial 의 정의로부터 $\phi_A(t) = \det(\lambda I-A)$ 를 계산한다. $$\begin{aligned} \left|\begin{matrix}t& -4&0&2\\0&t-2&0&0\\0&-2&t-1&0\\-1&0&-1&t-3\end{matrix}\right| &= t \left|\begin{matrix}t-2&0&0\\-2&t-1&0\\0&-1&t-3\end{matrix}\right| - (-1) \left|\begin{matrix}-4&0&2\\t-2&0&0\\-2&t-1&0\end{matrix}\right| \\&=t^4-6t^3+13t^2-12t+4 = (t-1)^2(t-2)^2\end{aligned}$$ 그리고 minimal polynomial 은 $t-1, t-2$ 를 인수로 가져야 하므로, $(t-1)(t-2)$ 로 놓고 계산을 해보면 $0$ 이 아니다. 인수를 하나씩 키워가며 계산을 해보면 $m_A(t) = (t-1)^2(t-2)$ 이 됨을 알 수 있다.
		\item[\textbf{(나)}] Minimal polynoimal 이 중근을 가지므로 diagonalizable 하지 않다.
		\item[\textbf{(다)}] $F^4=\ker(A-2I)^2 \oplus \ker (A-I)^2$ 으로 decompose 할 수 있다. 직접 basis 를 계산해 보면, $$(A-2I)^2=\begin{pmatrix}2&-8&-2&2\\0&0&0&0\\0&2&1&0\\-1&2&0&-1\end{pmatrix} \Rightarrow \ker(A-2I)^2 = \langle(1, 1, -2, 1)\trans, (1, 0, 0, -1)\trans\rangle$$
		$$(A-I)^2=\begin{pmatrix}-1&0&-2&-2\\0&1&0&0\\0&-2&0&0\\1&2&2&2\end{pmatrix} \Rightarrow \ker(A-I)^2 = \langle(0, 0, 1, -1)\trans, (2, 0, -1, 0)\trans\rangle$$ 을 얻는다. 이제 각 basis 의 vector 들에 $A$를 곱하면 차례대로 $\ker(A-2I)^2$ 의 basis vector 들은 $(2, 2, -4, 2)\trans, (2, 0, 0, -2)\trans$, $\ker(A-I)^2$ 의 basis 들은 $(2, 0, 1, -2)\trans, (0, 0, -1, 1)\trans$ 가 되어 \\$$\begin{aligned}(2, 2, -4, 2)\trans &= 2(1, 1, -2, 1)\trans \\ (2, 0,0, -2)\trans &= 2(1, 0, 0, -1)\trans \\ (2, 0, 1, -2)\trans &= 2(0, 0, 1, -1)\trans + (2, 0, -1, 0)\trans \\(0, 0, -1, 1)\trans &= -(0, 0, 1, -1)\trans\end{aligned}$$ 따라서 구하는 행렬 표현은 $$\begin{pmatrix}2&0&0&0\\0&2&0&0\\0&0&2&-1\\0&0&1&0\end{pmatrix}$$
	\end{itemize}
	
\item[\textbf{8.4.5}]
	\begin{itemize}
		\item[\textbf{(가)}] Block matrix 로 생각하자. $n\times 1$ vector $X, Y$를 생각하여 다음과 같이 두자. $$\begin{pmatrix}0&I_n\\I_n&0\end{pmatrix}\begin{pmatrix}X\\Y\end{pmatrix} = \lambda \begin{pmatrix}X\\Y\end{pmatrix}$$ 그러면 $Y=\lambda X, X=\lambda Y$ 로부터 가능한 $\lambda$의 값은 $\pm 1$ 뿐이다. 각 eigenvalue 에 대하여 일차독립인 eigenvector 를 $n$개 잡을 수 있으므로, ($X$를 $F^n$의 basis $\mathrm{\mathbf{e}}_1, \cdots, \mathrm{\mathbf{e}}_n$로 두면 $Y$가 자동으로 결정된다) $\dim E_1 = n, \dim E_{-1} = n$ 이라는 결론을 얻는다. 이로부터 eigenspace decomposition $F^{2n}=E_1 \oplus E_{-1}$ 을 얻으므로 (관찰 7.6.4) $A$는 diagonalizable 이며, $A \sim \text{diag}(1, \cdots, 1, -1, \cdots, -1)$ ($1, -1$이 각각 $n$개).
		\item[\textbf{(나)}] 대각행렬의 characteristic polynomial 과 minimal polynomial 은 구하기 쉽다. $\phi_A(t) = (t-1)^n(t+1)^n, m_A(t) = (t-1)(t+1)$ 임을 알 수 있다. (diagonalizable 이므로 $t-1, t+1$의 지수는 1 이어야 한다).
	\end{itemize}
	
\item[\textbf{8.4.6}] 
	\begin{itemize}
		\item[\textbf{(가)}] 우선 $L$의 eigenvalue 를 찾아보자. $LX=\lambda X$, 즉 $X+X\trans = \lambda X$ 인 $\lambda$ 를 찾아주면 된다. By inspection, $X$가 symmetric 일 때와 skew-symmetric 일 때 확인해 준다. \\ 먼저 $X$가 symmetric 일 경우에는 $X=X\trans$ 이므로 $2X=\lambda X$가 되어 가능한 eigenvalue 는 $2$ 뿐이다. 이 eigenvalue 에 대하여 가능한 eigenvector 는 symmetric matrices 이고, $\dim \mathfrak{Sym}_n(F)= n(n+1)/2$ 임을 알고 있으므로 $n(n+1)/2$ 개의 eigenvector 들을 잡아줄 수 있다. \\ $X$가 skew-symmetric 일 경우에는 $X=-X\trans$ 이므로 $X+X\trans = X - X = 0X = \lambda X$ 가 되어 가능한 eigenvalue 는 $0$ 이다. 이 eigenvalue 에 대하여 가능한 eigenvector 는 skew-symmetric matrices 이고, $\dim \mathfrak{Alt}_n(F)=n(n-1)/2$ 임을 알고 있으므로, $n(n-1)/2$ 개의 eigenvector 들을 잡아줄 수 있다.\\
		$\dim \mathfrak{Sym}_n(F) + \dim \mathfrak{Alt}_n(F) = n^2 = \dim \matrixM_{n, n}(F)$ 인 것과, $\mathfrak{Sym}_n(F) \cap \mathfrak{Alt}_n(F) = 0$ 으로부터, 관찰 7.6.4에 의해 우리는 eigenspace decomposition $\matrixM_{n, n}(F) = E_0  \oplus E_2$ 를 얻는다. 따라서 $L$은 diagonalizable.
		\item[\textbf{(나)}] (가)의 결과로부터 $L$을 대각화 하면 $L \sim \text{diag}(2, \cdots, 2, 0, \cdots, 0)$ 임을 알 수 있다. (2 는 $n(n+1)/2$ 개, 0 은 $n(n-1)/2$ 개) 그리고 diagonalizable 이므로 minimal polynomial 은 일차식의 곱이며 중근이 없어야 한다. 따라서 $$\phi_L(t) = t^{\frac{n(n-1)}{2}}(t-2)^{\frac{n(n+1)}{2}}, \quad m_L(t) = t(t-2)$$
	\end{itemize}
	
\item[\textbf{8.4.8}]
	\begin{itemize}
		\item[\textbf{(가)}] 우선 $T^3=T$ 로부터 $t^3-t \in \mathcal{I}_T$ 임을 알 수 있다. $m_T(t)$는 $t^3-t$ 의 약수여야 하는데, $t^3-t = t(t+1)(t-1)$ 으로부터 $t^3-t$ 의 약수들은 일차식의 곱으로 인수분해 되며 중근을 갖지 않는다는 것을 알 수 있다. 따라서 $T$는 diagonalizable.
		\item[\textbf{(나)}] Dimension theorem 으로부터 $\dim V = \dim \im T + \dim\ker T$ 이다. \\이제 $\im T \cap \ker T = 0$ 을 보이자. $v \in \im T \cap \ker T$ 라고 할 때, $Tu = v$ 인 $u\in V$ 가 존재하며, $Tv= 0$ 이다. $T^2u = Tv = 0$ 이므로, $$0 = T(0) = T(T^2u) = T^3u = Tu = v$$ 가 되어 $\im T \cap \ker T$ 의 원소들은 전부 $0$ 이다. 따라서 $V = \im T \oplus \ker T$. \\ 이제 $\im T = E_1 \oplus E_{-1}$ 임을 보이자. $v\in \im T$ 이면, $v = T^2v + (v - T^2v)$ 로 쓸 수 있으므로, $V = \im T \oplus \ker T$ 로부터 $v - T^2v = 0$ 이어야만 한다. ($T(v-T^2v) = 0$ 이므로 $v-T^2v\in \ker T$). $T$의 $\im T$ 위로의 restriction $U$를 고려하면, $\forall v\in \im T$ 에 대하여 $v-T^2v=0$ 이므로 $U^2=I$ 이다. $Uv=\lambda v$ 라고 할 때, $U^2v = \lambda Uv = \lambda^2v$ 이므로 가능한 eigenvalue 는 $\pm 1$ 뿐이다. 또 $U$의 minimal polynomial 이 ($t^2-1$ 의 약수) 일차식의 곱이고 중근이 존재할 수 없으므로, $U$ 는 diagonalizable 이며 eigenspace decomposition 으로 $\im T = E_1 \oplus E_{-1}$ 을 갖는다.
	\end{itemize}
	
\item[\textbf{8.5.10}] 
	\begin{itemize}
		\item[\textbf{(가)}] 우선 $f(T)=0$ 인 다항식 $f(t)$ 는 존재하므로, $\mathcal{I}_w \neq \{0\}$. 이제 $m_w(t)$가 최저 차수의 monic polynomial 임을 보여야 한다. $m_w(t) = m_{T|_W}(t)$ 이므로  $m_{T|_W}(t)$ 를 살펴보자. ($W = F[t]w$) $m_{T|_W}(t)$ 는 $\mathcal{I}_{T|_W}$ 의 원소들 중에서 최소의 degree 를 갖는 monic polynomial 이므로, $m_w(t)$ 가 monic 임은 당연하다. $\mathcal{I}_{T|_W}$ 의 정의는 $\{g(t)\in F[t] \: | \:g(T|_W) = 0\}$ 이고 $W$가 $T$-invariant subspace 이므로 $g(T|_W) = g(T)|_W$ (연습문제 8.2.12) 으로부터 $\mathcal{I}_{T|_W} = \{g(t)\in F[t] \: | \:g(T)|_W = 0\}$. $w$에 대하여 $g(T)|_Ww = 0$ 이므로 $\mathcal{I}_{T|_W} \subseteq \mathcal{I}_w$. \\이제 $T$-cyclic subspace of $V$ generated by $w$, $F[t]w$ 를 고려하자. $W=F[t]w$ 는 $(T|_W)$-cyclic 이며 $f(t)\in F[t]$ 에 대해 $f(T|_W)w=0$ 이면 $f(T|_W)=0$ (연습문제 8.5.3) 이므로 $\mathcal{I}_w \subseteq  \mathcal{I}_{T|_W}$. 따라서 $\mathcal{I}_w = \mathcal{I}_{T|_W}$. Minimal polynomial 의 정의로부터 우리가 원하는 결론을 얻는다.  
		\item[\textbf{(나)}] $f(t)$를 $m_w(t)$로 나는 몫을 $q(t)$, 나머지를 $r(t)$ 라고 하자. 그러면 $f(t)=m_w(t)q(t) + r(t)$ 이고 이를 $T|_W$에서 evaluate 하면, $$f(T|_W)=m_w(T|_W)q(T|_W)+r(T|_W) = r(T|_W)$$ 이므로 $0 = f(T|_W)w = r(T|_W)w = r(T)w$ 가 되어 $r(t)\in \mathcal{I}_w$ 이다. $m_w(t)$가 minimal polynomial 이라는 것에 모순되지 않으려면 $r(t) = 0$ 이어야 한다. 따라서 $f(t)$는 $m_w(t)$의 배수이다.
	\end{itemize}

\item[\textbf{8.5.14}] 연습문제 8.2.4의 $\{f(T)v\in F^2\: |\: f(t)\in F[t]\} = \langle v, Tv, T^2v, \cdots\rangle$ 를 이용한다. \\$A = I_2$ 일 때,
	\begin{itemize}
		\item[\textbf{(가)}] $\langle v, Tv, \cdots \rangle = \langle v\rangle$ 이고, 임의의 $v\in F^2$ 에 대하여 $\langle v\rangle = F^2$ 이게 할 수 없다.
		\item[\textbf{(나)}] $\langle v, Tv, \cdots \rangle = \langle v\rangle$ 을 $W$로 두면, $A(W) = W$ 이므로 $W$는 $A$-invariant subspace 이고, $W$의 정의로부터 $A$-cyclic 이다. 따라서 $W$는 $A$-cyclic subspace of $V$ generated by $v$. $U_1 = \langle (1, 0)\trans \rangle, U_2 = \langle (0, 1)\trans \rangle$ 으로 둘 때, $F^2=U_1 \oplus U_2$ 이다. (표준단위벡터를 생각)
	\end{itemize} 
	$A = \begin{pmatrix}1&0\\0&2\end{pmatrix}$ 일 때,
	\begin{itemize}
		\item[\textbf{(가)}] 임의의 $(a, b)\trans\in F^2$ 에 대하여, $v=(1, 1)\trans$ 로 잡으면 $f(t) = (b-a)t+2a-b \in F[t]$ 에 대하여 $f(T)v = (a, b)\trans \in F^2$ 가 되어 $F^2$는 $A$-cyclic space 이다.
		\item[\textbf{(나)}] $U_1 = \langle (1, 0)\trans \rangle, U_2 = \langle (0, 1)\trans \rangle$ 으로 둘 때, $F^2=U_1 \oplus U_2$ 이다. $U_1, U_2$ 모두 $A$-cyclic subspace 인 것은 계산을 통해 확인할 수 있다.
	\end{itemize}
	$A = \begin{pmatrix}1&1\\0&1\end{pmatrix}$ 일 때,
	\begin{itemize}
		\item[\textbf{(가)}] 임의의 $(a, b)\trans\in F^2$ 에 대하여, $v=(1, 1)\trans$ 로 잡으면 $f(t) = (a-b)t+2b-a \in F[t]$ 에 대하여 $f(T)v = (a, b)\trans \in F^2$ 가 되어 $F^2$는 $A$-cyclic space 이다.
		\item[\textbf{(나)}] $\phi_A(t) = m_A(t) = (t-1)^2$ 이므로, cyclic decomposition theorem 에 의하면, $2 = f=r_1\geq r_2\geq \cdots \geq r_h = 1$ 이고 $r_1+r_2+\cdots+r_h=e=2$ 이므로 $r_1=2$ 로만 decompose 가능하다. $F^2 = \langle(1, 0)\trans, (0, 1)\trans\rangle$ 로 쓸 수 있다.
	\end{itemize}

\item[\textbf{8.7.6}]
	\begin{itemize}
		\item[\textbf{(가)}] $N = \mathrm{J}_{(e)}-\lambda I_e$ 으로부터, $N$은 $e\times e$ matrix 이고,
		$$N = \begin{pmatrix} 0 & 1  \\ &0&1&&&\mathbf{0}\\&&0&1 \\ &&&\cdot&\cdot\\&&&&\cdot&\cdot \\ &\mathbf{0}&&&&0&1\\&&&&&&0  \end{pmatrix}$$ 이고, 실제로 거듭제곱을 해 보면, $1$ 들이 오른쪽 위로 평행이동 됨을 확인할 수 있다. $N^k = (a_{ij})$, $a_{ij} = 1$ if $j = i+k$, $0$ otherwise. 이를 귀납법으로 보이자. $k = 1$ 일 때에는 자명하다. $k$ 일때 성립함을 가정하고 $N^{k+1} = (c_{ij})$ 을 계산해 보자. $N^k=(a_{ij})$ 라고 하고, $N=(b_{ij})$ 로 두자. \\
		$c_{ij} = \sum_{x=1}^{e} a_{ix}b_{xj}$ 인데, $j = i+k+1$ 인 경우, $x=i+k\: (<e)$ 일 때만 summand 가 $1$ 이고 나머지 $x$ 에 대해서는 $0$ 이다. $j \neq i+k+1$ 인 경우에는 $x=i+k\: (<e)$ 일 때에도 $b_{xj} = 0$ 이 되어 결과적으로 $0$ 이다. 따라서 $k+1$ 인 경우에도 성립한다. 
		\item[\textbf{(나)}] $\big(\mathrm{J}_{(e)}\big)^m = (\lambda I_e+N)^m = \sum_{r= 0}^m {m\choose r}(\lambda I_e)^{m-r}N^r=\sum_{r=0}^m {m\choose r}\lambda^{m-r}N^r$ 이므로 ($N^0=I$ 로 이해한다), 따라서 $$\big(\mathrm{J}_{(e)}\big)^m = \begin{pmatrix}\lambda^m&m\lambda^{m-1}&{m \choose 2} \lambda^{m-2}&{m \choose 3} \lambda^{m-3}&\cdots&{m \choose r-1} \lambda^{m-r+1} \\ &\lambda^m&m\lambda^{m-1}&{m \choose 2} \lambda^{m-2}&\cdots&{m \choose r-2} \lambda^{m-r+2}\\ &&\ddots&\ddots&\ddots&\vdots\\&\mathbf{0}&&\ddots&\ddots&\vdots\\&&&&\lambda^m&m\lambda^{m-1}\\&&&&&\lambda^m\end{pmatrix}$$ 이고, $r-1 > m$ 인 경우에는 binomal coefficient 를 0으로 이해한다.
	\end{itemize}

\item[\textbf{8.7.7}] 
	\begin{itemize}
		\item[\textbf{(가)}] Block diagonal matrix 의 charateristic polynomial 은 block 들의 characteristic polynomial 의 곱이며, minimal polynomial 은 각 block 의 minimal polynomial 의 최소공배수 임을 이용한다. \\
		$\begin{pmatrix}\lambda&1&\\&\lambda&1\\&&\lambda\end{pmatrix}, \begin{pmatrix}\lambda&1\\&\lambda\end{pmatrix}, \begin{pmatrix}\lambda\end{pmatrix}$ 의 minimal polynomial 은 $(t-\lambda)^3, (t-\lambda)^2, t-\lambda$ 이므로, $\mathrm{J}_{(3, 3, 1)}, \mathrm{J}_{(3, 2, 2)}$ 의 minimal polynomial 은 $(t-\lambda)^3$ 이다. (Characteristic polynomial 은 $(t-\lambda)^7$) 이제 $\dim E_\lambda$ 를 계산해 보자. 각 행렬에서 $\lambda I_7$ 을 빼 보면 $1$의 개수가 $4$ 임을 알 수 있다. 따라서 두 행렬 모두 $\dim E_\lambda= 4$. 두 행렬이 similar 하지 않은 것은 (나)의 결과로부터 알 수 있다.
		\item[\textbf{(나)}] Dimension theorem 으로부터, $7 = \dim \im(A-\lambda I)^2 + \dim \ker (A-\lambda I)^2$, $\dim \ker (A-\lambda I)^2 = 7 - \mathrm{rk} (A-\lambda I)^2$. $\mathrm{J}_{(3, 3, 1)}$ 에 대해 $(\mathrm{J}_{(3, 3, 1)}-\lambda I)^2$ 를 계산해 보면, $(1, 3), (4, 6)$-성분만 1 이다. 일차독립인 row 의 개수는 2개 이므로 $\mathrm{J}_{(3, 3, 1)}$ 의 경우에는 5. $\mathrm{J}_{(3, 2, 2)}$ 에 대해 $(\mathrm{J}_{(3, 2, 2)}-\lambda I)^2$ 를 계산해 보면, $(1, 3)$-성분만 1 이다. 일차독립인 row 의 개수는 1개 이므로 $\mathrm{J}_{(3, 2, 2)}$ 의 경우에는 6. (따라서 두 행렬은 similar 하지 않다.)
	\end{itemize}              
\end{itemize}
\end{document}