%!TEX encoding = utf-8
\documentclass{article}
\usepackage{kotex}
\usepackage{amsmath}
\usepackage{amsfonts}
\usepackage{amssymb}
\usepackage{mathtools}
\usepackage{geometry}
	\geometry{
		top = 30mm,
		left = 30mm,
		right = 30mm,
	}

\pagenumbering{gobble}
\renewcommand{\baselinestretch}{1.3}
\newcommand*{\trans}{^{\mathrm{\mathbf{t}}}}%
\newcommand*{\basisB}{\mathfrak{B}}%
\newcommand*{\basisC}{\mathfrak{C}}%
\newcommand*{\basisD}{\mathfrak{D}}%
\newcommand*{\matrixM}{\mathfrak{M}}%

\begin{document}
\begin{center}
\textbf{\Large선형대수학 과제 3}\\
\large 2017-18570 이성찬
\end{center}

\begin{itemize}
\item[\textbf{7.2.4}] 
	\begin{itemize}
		\item[\textbf{(가)}] $A$가 nilpotent이고 diagonalizable 이므로, $A=UDU^{-1}$ 인 diagonal matrix $D \in \matrixM_{n, n}(F)$,  invertible matrix $U\in \matrixM_{n, n}(F)$가 존재하고, $A^r = 0$ 인 $r\geq 1$이 존재한다. $$A^r = (UDU^{-1})^r = UD^rU^{-1} = 0$$ 이고 $D^r = 0$ 이 된다. 그런데 diagonal matrix의 거듭제곱은 그 대각성분들의 거듭제곱과 같으므로, $D$의 모든 대각 성분들은 $0$이어야 한다. 그러므로 $D=0$ 이 되어 $A=0$ 이라는 결론을 얻을 수 있다.
		\item[\textbf{(나)}] $A$가 unipotent 이므로 $A-I$가 nilpotent 이다. 이제 $A-I$가 diagonalizable 임을 보이면, (가)에 의해 $A-I=0$ 이므로 $A=I$임을 보일 수 있다. $A$가 diagonalizable 이므로 $A=UDU^{-1}$ 인 diagonal matrix $D \in \matrixM_{n, n}(F)$,  invertible matrix $U\in \matrixM_{n, n}(F)$가 존재한다. $$A-I = UDU^{-1} - I = U(D - I)U^{-1}$$ 이 되고 $D-I$는 당연히 diagonal matrix 이므로 $A-I$는 diagonalizable matrix 이다. 따라서 (가)에 의해 $A=I$.
	\end{itemize}

\item[\textbf{7.2.5}]
	\begin{itemize}
		\item[\textbf{(가)}] Eigenvalue 가 $\lambda^2 - 1 = 0$ 으로부터 $\lambda = \pm 1$이므로 
				$D = \begin{pmatrix}
						1 & 0\\
						0 & -1\\
				\end{pmatrix}$ 이다. 
				Eigenvector를 구하면 $\lambda = 1$일 때 
				$(1, 1)\trans$, $\lambda = -1$일 때 $(1, -1)\trans$ 이고 서로 독립이므로 두 벡터는 $F^2$의 basis가 된다. 따라서
				$U = \begin{pmatrix}
						1 & 1\\
						1 & -1
				\end{pmatrix}$ 이다.
			
			$$\begin{pmatrix}
				1 & 1\\
				1 & -1\\
				\end{pmatrix}^{-1}
				\begin{pmatrix}
				0 & 1\\
				1 & 0\\
				\end{pmatrix} 
				\begin{pmatrix}
				1 & 1\\
				1 & -1\\
				\end{pmatrix}= 
				\begin{pmatrix}
				1 & 0\\
				0 & -1\\
				\end{pmatrix}
			$$
		\item[\textbf{(나)}] $\det(\lambda I - A) = \lambda^2-2\lambda-3 = (\lambda-3)(\lambda+1) = 0$ 이므로 eigenvalue는 $-1, 3$이다. 각각 대응하는 eigenvector를 구해주면 $-1$일 때 $(1, -1)\trans$, $3$일 때 $(1, 1)\trans$이다. \\따라서 $U = \begin{pmatrix}
				1 & 1\\
				-1 & 1\\
				\end{pmatrix}, D = \begin{pmatrix}
				-1 & 0\\
				0 & 3\\
				\end{pmatrix}$ 
				$$\begin{pmatrix}
				1 & 1\\
				-1 & 1\\
				\end{pmatrix}^{-1}
				\begin{pmatrix}
				1 & 2\\
				2 & 1\\
				\end{pmatrix}
				\begin{pmatrix}
				1 & 1\\
				-1 & 1\\
				\end{pmatrix}= 
				\begin{pmatrix}
				-1 & 0\\
				0 & 3\\
				\end{pmatrix}
			$$ 
		\item[\textbf{(다)}] $\det(\lambda I - A) = 0$ 으로부터 eigenvalue 는 $1, 1/4$이다. 각각 대응하는 eigenvector를 구해주면 $1$일 때 $(1, 1)\trans$, $1/4$일 때 $(-1, 2)\trans$이다. 따라서, $U = \begin{pmatrix}
				1 & -1\\
				1 & 2\\
				\end{pmatrix}, D = \begin{pmatrix}
				1 & 0\\
				0 & 1/4\\
				\end{pmatrix}$
				$$\begin{pmatrix}
				1 & -1\\
				1 & 2\\
				\end{pmatrix}^{-1}
				\begin{pmatrix}
				0.75 & 0.25\\
				0.5 & 0.5\\
				\end{pmatrix}
				\begin{pmatrix}
				1 & -1\\
				1 & 2\\
				\end{pmatrix} = 
				\begin{pmatrix}
				1 & 0\\
				0 & 0.25\\
				\end{pmatrix}$$
		
	\end{itemize}

\item[\textbf{7.2.12}]
	\begin{itemize}
		\item[\textbf{(가)}] $R_\theta$의 characteristic polynomial을 구하면 $\phi_{R_\theta}(t) = \det(tI-R_\theta) = t^2-2t\cos\theta + 1$이다. 그런데 $R_\theta \neq \pm I$이므로 $\theta \neq 0, \pi$. 즉 $\cos\theta \neq \pm 1$ 이다. 그러면 $\phi_{R_\theta}$의 판별식은 $D/4=\cos^2\theta-1$ 이고 $\cos^2\theta \neq 1$이므로 $D/4<0$이 되어 characteristic polynomial이 $\mathbb{R}$에서 실근을 갖지 않는다. 그림으로 설명하자면, $R_\theta v = \lambda v$인 $v\in \mathbb{R}^2$ 로부터 - $v$를 $\theta$만큼 회전변환 했을 때, 그 결과가 $v$의 $\lambda$(상수)배가 되게하는 $\lambda$가 존재하는가? - 와 동치이다. 회전변환이므로 $v$의 길이는 변할 수 없다. 따라서 $|\lambda| = 1$이어야 한다. 그런데 $R_\theta \neq \pm I$ 이므로, 실수 범위 내에서는 불가능하다. 따라서 $\mathbb{R}$에서는 not diagonalizable.
		\item[\textbf{(나)}] $\mathbb{C}$에서 $t^2-2t\cos\theta + 1=0$의 해를 구해보면, $t = \cos\theta \pm i\sin\theta$ 이다. 각 eigenvalue에 대응하는 eigenvector를 구해주면 $\cos\theta + i\sin\theta$일 때 $(i, 1)\trans$, $\cos\theta - i\sin\theta$일 때 $(-i, 1)\trans$이 되어 대각화 가능하다.
		$$\begin{pmatrix}
				\cos\theta & -\sin\theta\\
				\sin\theta & \cos\theta\\
				\end{pmatrix} = 
				\begin{pmatrix}
				i & -i\\
				1 & 1\\
				\end{pmatrix}
				\begin{pmatrix}
				\cos\theta+i\sin\theta & 0\\
				0 & \cos\theta-i\sin\theta\\
				\end{pmatrix}
				\begin{pmatrix}
				i & -i\\
				1 & 1\\
				\end{pmatrix}^{-1}$$
	\end{itemize}

\item[\textbf{7.2.22}] $\phi_A(t)=t^2-bt-a$ 의 서로 다른 두 근 $\lambda, \mu$에 대하여, 각각 대응하는 eigenvector를 구해주면 $U = \begin{pmatrix}1&1\\\lambda&\mu\end{pmatrix}$ 가 됨을 알 수 있고, $\begin{pmatrix}0&1\\a&b\end{pmatrix}^n \begin{pmatrix}x_1\\x_2\end{pmatrix} = \begin{pmatrix}x_{n+1}\\x_{n+2}\end{pmatrix}$ 로부터 $A^n$에 $U\cdot \text{diag}(\lambda^n, \mu^n)\cdot U^{-1}$  대입해 주면,
$$\begin{pmatrix}x_{n+1}\\x_{n+2}\end{pmatrix} = \begin{pmatrix}1&1\\\lambda&\mu\end{pmatrix} \begin{pmatrix}\lambda^n&0\\0&\mu^n\end{pmatrix}\begin{pmatrix}1&1\\\lambda&\mu\end{pmatrix}^{-1} \begin{pmatrix}x_1\\x_2\end{pmatrix}= \frac{1}{\mu-\lambda}\begin{pmatrix}1&1\\\lambda&\mu\end{pmatrix} \begin{pmatrix}\lambda^n&0\\0&\mu^n\end{pmatrix}\begin{pmatrix}\mu&-1\\-\lambda&1\end{pmatrix}\begin{pmatrix}x_1\\x_2\end{pmatrix}$$ 를 얻는다. 주어진 Fibonacci Sequence 에서는 $a=b=1$ 이고 $\lambda = (1+\sqrt{5})/2, \mu = (1-\sqrt{5})/2$ 이므로 대입해 계산해 주면,
$$\begin{pmatrix}x_{n+1}\\x_{n+2}\end{pmatrix} = \frac{1}{\mu-\lambda} \begin{pmatrix}\mu \lambda^n-\lambda \mu^n & -\lambda^n+\mu^n \\ \mu \lambda^{n+1}-\lambda \mu^{n+1} & -\lambda^{n+1}+\mu^{n+1}\end{pmatrix}\begin{pmatrix}x_1\\x_2\end{pmatrix}$$
따라서 $x_{n+2} = -\dfrac{1}{\sqrt{5}}\{\mu\lambda (\lambda^n-\mu^n) x_1 + (-\lambda^{n+1} + \mu^{n+1}) x_2 \} = \dfrac{1}{\sqrt{5}}\{(\lambda^n-\mu^n) x_1 + (\lambda^{n+1} - \mu^{n+1}) x_2 \}$.

\item[\textbf{7.3.8}]
	\begin{itemize}
		\item[\textbf{(가)}] ($\Rightarrow$) $T\in\mathfrak{LM}$ 이 nilpotent 이면 따름정리 7.3.6의 증명에 의해 모든 eigenvalue 가 $0$이 된다.\\ ($\Leftarrow$) $T\in\mathfrak{LM}$ 의 모든 eigenvalue 가 $0$ 이면, $F=\mathbb{C}$ 이므로 $T$를 triangularize 하면 $T = UDU^{-1}$ 인 upper-triangular matrix $D$와 가역 사상 $U$가 존재하며, $D$의 대각 성분들은 전부 0이 된다. 그러므로 $D$는 strictly upper-triangular 이다. 이제 연습문제 1.1.7 (다)에 의하면, $D$가 nilpotent 이다. $D^r = 0$ for some $r\geq 1$ 이라고 하면, $T^r = (UDU^{-1})^r = UD^rU^{-1} = 0$ 이 되어 $T$가 nilpotent 이다. 
		\item[\textbf{(나)}] ($\Rightarrow$) $T$가 unipotent 이면 $T-I$가 nilpotent 이다. $T$를 $\mathbb{C}$ 위에서 triangularize 하면, $T-I$의 eigenvalue 는 전부 0 임을 알 수 있다. (따름정리 7.3.6) $T-I$와 similar 한 strictly upper-triangular matrix가 존재하며, $T-I = UDU^{-1}$ (단, $U$는 가역) 라고 하면, $T = UDU^{-1} + I = U(D+I)U^{-1}$이 되고, $D+I$ 는 upper-triangular 이며, 대각성분이 전부 $1$ 이다. 따라서 $D+I$의 모든 eigenvalue 는 1 이다. 서로 similar 한 matrix 는 characteristic polynomial 이 같으므로 eigenvalue 도 같다. $T \sim (D+I)$ 이므로 $T$의 eigenvalue 도 전부 1 이다. 
		\\ ($\Leftarrow$) $T$의 모든 eigenvalue 가 1 이면, $T$를 triangularize 하여 $T = UDU^{-1}$인 대각 성분이 전부 1인 upper-triangular matrix $D$와 가역인 U가 존재한다. 이제 $D = D'+I$ 로 쓰면, ($D'$은 strictly upper-triangular) $T-I = UD'U^{-1}$ 이고 $D'$은 nilpotent 이므로 $D'^r = 0$ for some $r\geq 1$ 이라 하면, $(T-I)^r = 0$이 되어 $T-I$는 unipotent 이다.
	\end{itemize}

\item[\textbf{7.4.2}] 연습문제 1.1.20 (가)에 의해 $T, U\in \mathfrak{LM}$ 이고 $U$가 가역일 때, $m\geq 0$이면, $(U^{-1}TU)^m = U^{-1}T^mU$ 임을 적극적으로 이용한다.
$f(t) = a_nt^n + \cdots + a_1t+a_0$ (단, $a_n, \cdots, a_1, a_0 \in F)$ 라고 하면,  \\$\begin{aligned}
f(U^{-1}TU) &= a_n(U^{-1}TU)^n + \cdots + a_1(U^{-1}TU) + a_0I \\
	&= a_n(U^{-1}T^nU) + \cdots + a_1(U^{-1}TU) + a_0(U^{-1}IU)\\
	&= U^{-1}(a_nT^n + \cdots + a_1T + a_0I)U = U^{-1}f(T)U
\end{aligned}$\\\\
이다. 따라서 $T \sim S$ 이면, $T = USU^{-1}$인 가역 사상 $U$가 존재하므로, $f(T) = f(USU^{-1}) = Uf(S)U^{-1}$ 이므로 $f(T) \sim f(S)$.
 
\item[\textbf{7.5.8}] 주어진 세 행렬들을 차례대로 $A, B, C$라고 두자. 각각 characteristic polynomial을 구하면, $\phi_A(t) = \phi_B(t) = \phi_C(t) =  (t-1)^3$로 전부 같다. characteristic polynomial 만으로 판정이 되지 않으므로, minimal polynomial 을 구해보자. 이를 위해 $(t-1)^3$의 약수만을 조사해주면 된다. (관찰 7.5.4 (나)). \\ $m_A(t) = t-1$ 인 것은 당연하고, $B$의 경우에는 $B-I$가 $0$이 아니므로, $m_B(t) = (t-1)^2$이 됨을 확인할 수 있다. 마지막으로 $C$는 $C-I \neq 0, (C-I)^2 \neq 0$ 이므로 $m_C(t) = (t-1)^3$이다. (직접 계산을 통해 확인할 수 있다.) 관찰 7.5.7의 대우를 생각하면, minimal polynomial이 모두 다르므로, 세 행렬은 similar 하지 않다.

\item[\textbf{7.5.10}]
	\begin{itemize}
		\item[\textbf{(가)}] $\phi_A(t) = \det(tI - A) = \det
		\begin{pmatrix}
			tI - B & -D \\
			0 & tI-C
		\end{pmatrix} = \det(tI-B)\det(tI-C) = \phi_B(t)\phi_C(t)$. (6.5.9 (가)를 이용한다)
		\item[\textbf{(나)}] 연습문제 2.3.14 (나)를 이용하자. $0 = m_A(A) = \begin{pmatrix}m_A(B)& m_A(D) \\ 0 & m_A(C)\end{pmatrix}$ 이므로,\\ $m_A(B) = m_A(C) = m_A(D) = 0$ 이다. 따라서 $m_A(t)$는 $\mathcal{I}_B, \mathcal{I}_C$ 의 원소이다. 관찰 7.5.3에 의해 $m_A(t)$는 $m_B(t)$와 $m_C(t)$의 배수가 되어, $m_A(t)$는 $m_B(t)$와 $m_C(t)$의 공배수이다. $m_B(t), m_C(t)$의 최소공배수의 배수는 곧 $m_B(t), m_C(t)$의 공배수이므로, 우리가 원하는 결론을 얻는다.
		\item[\textbf{(다)}] (나)로부터 우리는 $m_A(t)$가 $\text{lcm}(m_B(t), m_C(t))$의 배수임을 알았다. 이제 $\text{lcm}(m_B(t), m_C(t))$를 $L(t)$로 두고 $D = 0$일 때, $L(A) = 0$을 보이면 $m_A(t) = \text{lcm}(m_B(t), m_C(t))$가 될 수밖에 없다. 
		$L(A) = \begin{pmatrix}L(B)& 0 \\ 0 & L(C)\end{pmatrix}$ 인데, $L(t)$는 각각 $m_B(t), m_C(t)$의 배수이므로, $L(B) = L(C) = 0$ 이다. 따라서 $L(A)=0$이 되어 $m_A(t) = \text{lcm}(m_B(t), m_C(t))$ 이다.
	\end{itemize}

\item[\textbf{7.5.12}] $T$가 unipotent 이므로 $(T-I)^n =0$ for some $n\geq 1$ 이라고 하자. 그러면 $f(t) = (t-1)^n \in \mathcal{I}_{T}$ 이고, $T^m=I$인 자연수 $m$이 존재하므로, $g(t) = t^m-1 \in \mathcal{I}_{T}$ 이다. 한편, $\mathcal{I}_T$의 모든 원소들은 $m_T(t)$의 배수이므로, $f(t), g(t)$의 공약수 중 $m_T(t)$가 반드시 존재해야 한다. $f(t), g(t)$의 공약수는 이 둘의 최대공약수의 약수이므로 최대공약수를 구하기 위해 $g(t)$를 인수분해 하면 $(t-1)(t^{m-1}+\cdots+1)$ 이므로 최대공약수는 $t-1$임을 알 수 있다. 따라서 $m_T(t)$는 $t-1$의 약수여야 한다. 그런데 minimal polynomial 의 정의로부터 $\text{deg}(m_T) \geq 1$ 이므로 $m_T(t) = t-1$이 되며, $m_T(T) = T - I = 0$에서 $T=I$라는 결론을 얻는다.

\item[\textbf{7.5.13}] $\phi_A(t) = \det(tI-A) = \det \begin{pmatrix}t&0&-1&0\\0&t&0&-1\\0&-1&t&0\\-1&0&0&t \end{pmatrix} = t\det{\begin{pmatrix}t&0&-1\\-1&t&0\\0&0&t\end{pmatrix}} - \det{\begin{pmatrix}0&t&-1\\0&-1&0\\-1&0&t\end{pmatrix}}\\ = t^4 - 1$ 이다. 이제 이를 인수분해하면 $(t-1)(t+1)(t-i)(t+i)$를 얻는다. 이 인수들을 적절히 택하여 $A$를 대입했을 때 $0$이 되는지 확인해 보면, 모두 택해야 가능함을 알수 있다. 따라서 $m_A(t) = t^4-1$.$F=\mathbb{R}$이면 diagonalizable 하지 않지만, $\mathbb{C}$에서는 다음과 같이 할 수 있다.
$$A= \begin{pmatrix}-0.5&-0.5&-0.5i&0.5i\\-0.5&-0.5&0.5i&-0.5i\\-0.5&0.5&0.5&0.5\\-0.5&0.5&-0.5&-0.5\end{pmatrix}\begin{pmatrix}1&0&0&0\\0&-1&0&0\\0&0&i&0\\0&0&0&-i\end{pmatrix}\begin{pmatrix}-0.5&-0.5&-0.5i&0.5i\\-0.5&-0.5&0.5i&-0.5i\\-0.5&0.5&0.5&0.5\\-0.5&0.5&-0.5&-0.5\end{pmatrix}^{-1}$$

\item[\textbf{7.6.24}]
	\begin{itemize}
		\item[\textbf{(가)}] $\phi_A(t) = \det(tI-A) = \det(tI-A)\trans = \det(tI-A\trans) = \phi_{A\trans}(t)$ (By 따름정리 6.3.7, 연습문제 1.1.8 (가).) 따라서 $A, A\trans$가 같은 eigenvalue 를 갖는 것은 당연하다.
		\item[\textbf{(나)}] 연습문제 1.1.8 (라) 항을 이용하면 $(A^n)\trans=(A\trans)^n$ 임을 알 수 있다. 정의에 의해 $m_A(A) = 0$ 이므로 양변을 transpose 하면, $(m_A(A))\trans = 0$ 이다. $m_A(t) = a_nt^n + \cdots +a_1t+a_0 \quad (a_i\in F)$ 이라고 두었을 때,$$\begin{aligned}0 = (m_A(A))\trans &= (a_nA^n + \cdots + a_1A+a_0I)\trans = (a_nA^n)\trans + \cdots + (a_1A)\trans + (a_0I)\trans \\&= a_n(A\trans)^n + \cdots + a_1A\trans + a_0I = m_A(A\trans)\end{aligned}$$ 가 되어 $A$의 minimal polynomial 은 $m_{A\trans}(t)$의 배수이다. 같은 방법으로 하면 $A\trans$의 minimal polynomial이 $m_A(t)$의 배수임을 보일 수 있다. 서로가 서로를 나누는 두 monic polynomial은 서로 같을 수밖에 없다. 따라서 $m_A(t)=m_{A\trans}(t)$.
		\item[\textbf{(다)}] Dimension Theorem 으로부터, $\dim E_\lambda^A = \dim\ker(A-\lambda I) = \dim L_{A-\lambda I}  - \dim \text{im}(A-\lambda I) = n-\dim \text{im}(A-\lambda I)$. 마찬가지로 $\dim E_\lambda^{A\trans} = \dim\ker(A\trans-\lambda I) = n-\dim \text{im}(A\trans-\lambda I)$. 이제 $A-\lambda I$와 $A\trans -\lambda I$의 rank 가 같음을 보이면 원하는 결론을 얻는다. 두 행렬은 서로 transpose 이고, $A-\lambda I$의 row rank는 $A\trans - \lambda I$의 column rank 이므로, Rank Theorem 으로부터 $A - \lambda I$와 $A\trans - \lambda I$의 rank는 같다. 따라서 $\dim E_\lambda^{A} = \dim E_\lambda^{A\trans}$.
			\end{itemize}              
\end{itemize}

\end{document}