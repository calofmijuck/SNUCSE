%!TEX encoding = utf-8
\documentclass[11pt]{report}
\usepackage{kotex}
\usepackage{amsmath}
\usepackage{amsfonts}
\usepackage{amssymb}
\usepackage{mathtools}
\usepackage{geometry}
\geometry{
	top = 20mm,
	left = 20mm,
	right = 20mm,
	bottom = 20mm
}
\geometry{a4paper}

\pagenumbering{gobble}
\renewcommand{\baselinestretch}{1.3}
\newcommand{\mf}[1]{\mathfrak{#1}}
\newcommand{\mc}[1]{\mathcal{#1}}
\newcommand{\bb}[1]{\mathbb{#1}}
\newcommand{\rmbf}[1]{\mathrm{\mathbf{#1}}}
\newcommand{\inv}{^{-1}}
\newcommand{\norm}[1]{\left\lVert#1\right\rVert}
\newcommand{\paren}[1]{\left( #1 \right)}
\renewcommand{\span}[1]{\left\langle #1 \right\rangle}
\newcommand{\adj}{\text{*}}
\newcommand{\ra}{\rightarrow}
\newcommand{\Ra}{\Rightarrow}
\newcommand{\abs}[1]{\left|#1\right|}
\newcommand{\ds}{\displaystyle}
\newcommand{\inte}{\mathrm{int}\,}
\newcommand{\imp}{\implies}
\newcommand{\bs}{\setminus}
\newcommand{\diam}{\text{diam}}
\newcommand{\dist}{\text{dist}}
\newcommand{\mimp}{$\implies$}
\newcommand{\miff}{$\iff$}
\newcommand{\R}{\mathbb{R}}
\newcommand{\N}{\mathbb{N}}
\newcommand{\Z}{\mathbb{Z}}
\newcommand{\Q}{\mathbb{Q}}
\newcommand{\C}{\mathbb{C}}
\renewcommand{\d}{\,d}
\newcommand{\lint}[1]{\underline{\int#1}}
\newcommand{\uint}[1]{\overline{\int#1}}

\usepackage{enumitem}
\setlist[enumerate,1]{label={\textbf{\sffamily\arabic*.}}}
\setlist[enumerate,2]{label={\textbf{\sffamily(\arabic*)}}}

\begin{document}
\begin{center}
\textbf{\Large 해석개론 및 연습 1 과제 \#7}\\
\large 2017-18570 컴퓨터공학부 이성찬
\end{center}
\begin{enumerate}
\item
\begin{enumerate}
	\item Let $P = \{a = x_0 < x_1 < \cdots  < x_n = b\}$, $Q = \{a = y_0 < y_1 < \cdots < y_m = b\}$. If $P\subset Q$, there exists a sequence $\span{k(i)}_{i=0}^n$ s.t. $k(0) = 0$ and $k(n) = m$, where $y_{k(i)} = x_i$. Let $X = \{0,1, \dots, n\}, Y = \{0, 1, \dots, m\}$. Then
	$$\begin{aligned}
	V(f, P) &= \sum_{i=1}^{n} \abs{f(x_{i})-f(x_{i-1})}=\sum_{i=1}^n \abs{f(y_{k(i)}) - f(y_{k(i-1)})}\\
	&\leq \sum_{i\in X} \abs{f(y_{k(i)}) - f(y_{k(i-1)})} + \sum_{j \in Y\bs k(X)} \abs{f(y_j) - f(y_{j-1})}\\
	&=\sum_{i=1}^n \abs{f(y_i) - f(y_{i-1})} = V(f, Q)
	\end{aligned}$$
	\item By definition of $V(f)$, for any $\epsilon > 0$, there exists $P_0\in \mc{P}[a, b]$ s.t.
	$$V(f)-\epsilon < V(f, P_0)$$
	and by (1), if $P\supset P_0$, $V(f, P_0) \leq V(f, P)$. Also, it is trivial that $V(f, P)\leq V(f) < V(f) + \epsilon$. Thus we have the desired result,
	$$\abs{V(f, P) - V(f)} < \epsilon$$
\end{enumerate}

\item 
\begin{enumerate}
	\item \textbf{False}. Consider $$f(x) = \begin{cases}
	 	0 & (a \leq x \leq 0) \\ 1 & (0 < x \leq b)
	\end{cases}$$
	Then for $a\leq x\leq 0$, $F(x) = 0$.\\
	For $0 < x < \delta$, $P\in \mc{P}[a, x]$, define $P = \{a = x_0 < x_1 < \cdots < x_{l-1} \leq 0 < x_l < \cdots < x_n =x\}$. Then $V_a^x(f, P) = \abs{f(x_l)-f(0)} = 1$. Thus for $\epsilon = 1/2$, for any $\delta > 0$, there exists $x$ s.t. $\abs{x} < \delta$ and $ \abs{V_a^x(f, P)} \geq \epsilon $. $F(x)$ is discontinuous at $x = 0$.
	\item At $x_0 \in X = [a, b]$, for any $\epsilon > 0$, $x < x_0$, there exists $\delta >0$ s.t. $\abs{x-x_0} < \delta, x\in X \imp \abs{f(x) - f(x_0)} < \epsilon/2$.
	$F(x_0)-F(x) = V_x^{x_0}(f)$ since $f$ is a function of bounded variation. Take $y \in (x_0-\delta, x_0)$ and consider a partition $P\in\mc{P}[y, x_0]$. For $y < x < x_0$,
	$$V_y^{x_0}(f) < V_y^{x_0}(f, P) + \frac{\epsilon}{2}$$
	$$V_y^x(f) \geq V_y^{x_0}(f, P) - \abs{f(x_0) - f(x)} = V_y^x(f, P')$$
	($V_y^{x_0}(f, P) - \abs{f(x_0) - f(x)}$ is another variation on $[x, y]$ for a partition $P'$.) Combining these two gives
	$$V_x^{x_0}(f) = V_y^{x_0}(f) - V_y^x(f) < V_y^{x_0}(f, P) + \frac{\epsilon}{2} - V_y^{x_0}(f, P) + \abs{f(x)-f(x_0)} < \frac{\epsilon}{2}+\frac{\epsilon}{2} < \epsilon$$
	We can use a similar argument for $x > x_0$, and this proves that $F(x)$ is continuous.
\end{enumerate}

\item Let $P = \{a = x_0 < x_1 < \cdots  < x_n = b\}$.
$$V_a^b(f, P) = \sum_{i=1}^n \abs{f(x_i)-f(x_{i-1})} = \sum_{i=1}^n \abs{f'(t_i)}(x_i-x_{i-1}) \qquad t_i\in (x_{i-1}, x_i)$$
which is equal to $R(\abs{f'}, P)$. By definition, $R(\abs{f'}, P) = V_a^b(f, P) \leq V_a^b(f)$.
For all $\epsilon>0$, there exists some partition $P_1$ s.t.
$$V_a^b(f)-\epsilon < V_a^b(f, P_1) = R(\abs{f'}, P_1)$$
and if $P\supset P_1$, we have $$\abs{R(\abs{f'}, P) - V_a^b(f)} < \epsilon$$ and since $f'$ is integrable, $\abs{f'}$ is integrable, and $$V_a^b(f) = \int_a^b \abs{f'(t)}\,dt$$

\item $a > b > 0$.\\
Consider partition $$P = \left\{ \left(\frac{2}{(2n+1)\pi}\right)^{1/b} \right\}_{n=0}^\infty$$
Then $$V(f, P) = \left(\frac{2}{\pi}\right)^{a/b} + 2 \sum_{i=1}^\infty \left(\frac{2}{(2i+1)\pi}\right)^{a/b} < \infty \iff a > b > 0$$

\item \textsf{\textbf{(1\mimp2)}} $f\in \mc{R}(\alpha)$, $\int_a^b f\d\alpha = A$. There exists $P_1, P_2$ s.t.
$$U(f, P_1, \alpha) < A+\epsilon \qquad L(f, P_2, \alpha) > A- \epsilon$$
Setting $P_0 = P_1\cup P_2$, and if $P\supset P_0$,
$$A-\epsilon < L(f, P_0, \alpha) \leq L(f, P, \alpha) \leq S(f, P, \alpha) \leq U(f, P, \alpha) \leq U(f, P_0, \alpha) < A+ \epsilon$$
Thus we have
$$\abs{S(f, P, \alpha) - A} <\epsilon$$

\textsf{\textbf{(2\mimp1)}} For all $\epsilon > 0$, there exists $P_0$ s.t for all $P\supset P_0$, $$A - \frac{\epsilon}{3} < S(f, P, \alpha) < A+\frac{\epsilon}{3}$$
Take infimum on the left inequality, supremum on the right inequality to get
$$A-\frac{\epsilon}{3} \leq L(f, P, \alpha) \qquad U(f, P, \alpha) \leq A+\frac{\epsilon}{3}$$
Therefore
$$U(f, P, \alpha) - L(f, P, \alpha) < \frac{2\epsilon}{3} <\epsilon \imp f\in \mc{R}(\alpha)$$
Since
$$ A - \frac{\epsilon}{3} < L(f, P, \alpha) < \lint{_a^b} f\d\alpha \leq \uint{_a^b}f\d\alpha < U(f, P, \alpha) < A + \frac{\epsilon}{3}$$
setting $\epsilon\ra 0$ will give $\int_a^b f\d\alpha = A$ since $f\in \mc{R}(\alpha)$.


\item 
\begin{enumerate}
	\item $\alpha$ is monotone on $[0, 1]$ and $[1, 2]$. Therefore $\alpha$ is of bounded variation on each interval, thus $\alpha$ is of bounded variation on $[0, 2]$.
	\item Set
	$$\alpha_1(x) = \begin{cases}
	0& (x < 1) \\ 3 & (x\geq 1)
	\end{cases} \quad \alpha_2(x) = x^2 \quad \alpha_3(x) = \begin{cases}
		0 & (x < 1) \\ 2x^2 & (x \geq 1)
	\end{cases}$$
	so that $\alpha_i$ are increasing. (Also BV) Then $$\int_0^2 f\d\alpha = \int_0^2 f \d(\alpha_1 + \alpha_2) - \int_0^2 f\d\alpha_3 = \int_0^2 f \d\alpha_1 + \int_0^2 f \d\alpha_2 - \int_0^2 f\d\alpha_3$$
	where the last equality holds since $f$ is Stieltjes integrable w.r.t. $\alpha_1, \alpha_2$.
	Evaluating each integral gives
	$$\begin{aligned}
	\int_0^2 f\d\alpha &= \int_0^2 x^3 \d\alpha_1 + \int_0^2 x^3 \d(x^2) - \int_0^2 x^3 \d(2x^2) \\
	&=f(1) + \int_0^2 x^3\cdot 2xdx  - \int_1^2 x^3\cdot 4xdx\\
	&=1 + \frac{64}{5} - \frac{124}{5} = -11
	\end{aligned}$$
\end{enumerate}
\end{enumerate}
\end{document}