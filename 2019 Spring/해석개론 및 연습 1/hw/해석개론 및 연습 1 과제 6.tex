%!TEX encoding = utf-8
\documentclass[11pt]{report}
\usepackage{kotex}
\usepackage{amsmath}
\usepackage{amsfonts}
\usepackage{amssymb}
\usepackage{mathtools}
\usepackage{geometry}
\geometry{
	top = 20mm,
	left = 20mm,
	right = 20mm,
	bottom = 20mm
}
\geometry{a4paper}

\pagenumbering{gobble}
\renewcommand{\baselinestretch}{1.3}
\newcommand{\mf}[1]{\mathfrak{#1}}
\newcommand{\mc}[1]{\mathcal{#1}}
\newcommand{\bb}[1]{\mathbb{#1}}
\newcommand{\rmbf}[1]{\mathrm{\mathbf{#1}}}
\newcommand{\inv}{^{-1}}
\newcommand{\norm}[1]{\left\lVert#1\right\rVert}
\newcommand{\paren}[1]{\left( #1 \right)}
\renewcommand{\span}[1]{\left\langle #1 \right\rangle}
\newcommand{\adj}{\text{*}}
\newcommand{\ra}{\rightarrow}
\newcommand{\Ra}{\Rightarrow}
\newcommand{\abs}[1]{\left|#1\right|}
\newcommand{\ds}{\displaystyle}
\newcommand{\inte}{\mathrm{int}\,}
\newcommand{\imp}{\implies}
\newcommand{\bs}{\backslash}
\newcommand{\diam}{\text{diam}}
\newcommand{\dist}{\text{dist}}
\newcommand{\mimp}{$\implies$}
\newcommand{\miff}{$\iff$}
\newcommand{\R}{\mathbb{R}}
\newcommand{\N}{\mathbb{N}}
\newcommand{\Z}{\mathbb{Z}}
\newcommand{\Q}{\mathbb{Q}}
\newcommand{\C}{\mathbb{C}}

\usepackage{enumitem}
\setlist[enumerate,1]{label={\textbf{\sffamily\arabic*.}}}
\setlist[enumerate,2]{label={\textbf{\sffamily(\arabic*)}}}

\begin{document}
\begin{center}
\textbf{\Large 해석개론 및 연습 1 과제 \#6}\\
\large 2017-18570 컴퓨터공학부 이성찬
\end{center}
\begin{enumerate}
\item 
\begin{enumerate}
	\item Since $x^2$ is increasing, for the given partition,
	$$M_i = \frac{i}{n} \quad m_i = \frac{i-1}{n}$$
	thus 
	$$U(f, P_n) = \sum_{i=1}^{n} M_i (x_i-x_{i-1}) = \sum_{i=1}^n \left(\frac{i}{n}\right) \cdot \frac{1}{n} = \frac{n(n+1)(2n+1)}{6n^3}$$
	$$L(f, P_n) = \sum_{i=1}^{n} m_i (x_i-x_{i-1}) = \sum_{i=1}^n \left(\frac{i-1}{n}\right) \cdot \frac{1}{n} = \frac{n(n-1)(2n-1)}{6n^3}$$
	\item $$\lim_{n\ra\infty} U(f, P_n) = \lim_{n\ra\infty} \frac{1\cdot (1 + 1/n) \cdot (2 + 1/n)}{6} = \frac{1}{3}$$
	$$\lim_{n\ra\infty} L(f, P_n) = \lim_{n\ra\infty} \frac{1\cdot (1 - 1/n) \cdot (2 - 1/n)}{6} = \frac{1}{3}$$
\end{enumerate}

\item $f$ is not Riemann Integrable. For any partition $P = \{0 = x_0 < x_1 < \cdots <x_n = 1\}$,
$m_i = 0$ since there exists an irrational number in $[x_{i-1}, x_i]$. Also, $M_i=x_i$ since for any $\epsilon>0$, there exists a rational number in $(x_i-\epsilon, x_i)$. Thus $L(f, P) = 0$, and $$U(f, P) = \sum_{i=1}^n x_i(x_i-x_{i-1}) \geq \sum_{i=1}^n \frac{x_{i}+x_{i-1}}{2}(x_i-x_{i-1}) = \frac{1}{2} \sum_{i=1}^n (x_i^2 - x_{i-1}^2) = \frac{1}{2}x_n^2 = \frac{1}{2}$$
Therefore, $U(f, P) - L(f, P) = \frac{1}{2}$, and cannot be made arbitrarily small.

\item
\begin{enumerate}
	\item For any partition $P = \{a=x_0 < x_1<\cdots<x_n =b \} \in \mc{P}[a, b]$, $M_i = m_i =c$. $$U(f, P) = L(f, P) = \sum_{i=1}^n c(x_{i}-x_{i-1}) = c(b-a)$$
	$U(f, P), L(f, P)$ are always constant, so the upper/lower integrals also have the value $c(b-a)$ and thus $f$ is integrable.
	\item Let $Y = \{y_i : i = 1, \dots, n\}$ be the set of discontinuities of $f$. For any $\epsilon > 0$, set $$\delta = \min \left\{\frac{\epsilon}{2\sum_{j=1}^n \abs{c - f(y_j)}}, \min_{1< j\leq n}\left\{\frac{y_j-y_{j-1} }{4} \right\}, \frac{y_1-a}{4}, \frac{b-y_n}{4} \right\}$$
	Now consider $P = \{a < y_1 -\delta < y_1+\delta < \cdots < y_n-\delta < y_n+\delta < b \}$, and let each element be $x_0, \dots, x_{2n+2}$ in ascending order.
	$$U(f, P) - L(f, P) = \sum_{i=1}^{2n+2} (M_i - m_i)(x_i-x_{i-1}) = \sum_{j=1}^n \abs{c-f(y_j)} \cdot 2\delta < \epsilon$$
	the last equality holds because $M_i-m_i$ is $\abs{c - f(y_j)}$ for $(y_j-\delta, y_j+\delta)$, 0 otherwise.
	Thus $f$ is Riemann Integrable. Now we know that $\inf U(f, P)$ will equal $\int_a^b f$.
	$$\begin{aligned}
	U(f, P) &= \sum_{i=1}^{2n+2} M_i (x_i-x_{i-1}) \\&= c(y_1-\delta - a) + \sum_{j=1}^n \max\{c, f(y_i)\} \cdot 2\delta + c(b-y_n-\delta) + \sum_{j=2}^nc (y_i - y_{i-1}-2\delta)\\
	&=c(b-a) + c(y_1-y_n-2\delta) + c(y_n-y_1-2(n-1)\delta) + \sum_{j=1}^n \max\{c, f(y_i)\} \cdot 2\delta\\
	&=c(b-a) - 2nc\delta + \sum_{j=1}^n \max\{c, f(y_i)\} \cdot 2\delta \\&= c(b-a) + 2\delta \left( \sum_{j=1}^n \max\{c, f(y_i)\}- nc\right)\geq c(b-a)
	\end{aligned}$$
	The last inequality holds from $\max\{c, f(y_i)\} \geq c$. Now setting $\delta \ra 0$ (by setting $\epsilon \ra 0$) will give $\inf U(f, P) = c(b-a) = \int_a^b f$.
\end{enumerate}

\item We know that if $f, g$ are Riemann Integrable, $f-g$ is also Riemann Integrable. For any $P\in \mc{P}[a, b]$,
$$\begin{aligned}
	\int_a^b f &= \lim_{\norm{P}\ra0} \sum_{i=1}^n f(x_i^*)(x_i-x_{i-1}) \qquad (x_i^* \in [x_{i-1}, x_i]) \\
	&\geq \lim_{\norm{P}\ra0} \sum_{i=1}^n g(y_i^*)(x_i-x_{i-1}) \qquad (y_i^* \in [x_{i-1}, x_i])\\
	&= \int_a^b g
\end{aligned}
$$
This inequality works because $f, g$ are integrable and the Riemann sums exist.

\item 
\begin{enumerate}
	\item Consider
	$$f(x) = \begin{cases}
		1 & (x = x_0 = \frac{a+b}{2}) \\ 0 & (x \neq x_0)
	\end{cases}$$
	We know that $\int_a^b f = 0$ by \textbf{\sffamily 3 -- (2)}.
	\item Since $f(x)$ is continuous, there exists $\delta > 0$ s.t. $$x\in [a, b], \abs{x- x_0} <\delta \imp \abs{f(x)-f(x_0)} < \frac{f(x_0)}{2}$$
	Then in the interval $X = [a, b]\cap (x_0-\delta, x_0+ \delta)$, $\frac{f(x_0)}{2} < f(x)$.
	$$\int_a^b f = \int_X f + \int_{[a, b]\bs X} f \overset{(*)}{\geq}\int_X f > \frac{f(x_0)}{2} \min\{b-a, 2\delta\} > 0$$
	$(*)$: $f(x) \geq 0$ implies that the integral is greater than or equal to 0 in $[a, b]\bs X$. 
\end{enumerate}
\item 
\begin{enumerate}
	\item[\textbf{\sffamily (2)}] 
	Let $M = \sup\{f(t): t\in (a, b)\} - \inf\{f(t): t\in (a, b)\}$, $D_f = \{d_1, d_2, \dots \}$.  Let $\delta > 0$ and define $I \subset X=[a,b]$ as $$I = \left\{ x \in [a,b] : \exists\,\epsilon > 0 \text{ s.t. }\sup\{f(t): t\in N_X(x, \epsilon)\} - \inf\{f(t): t\in N_X(x, \epsilon)\} < \frac{\delta}{b-a}  \right\}$$  Then $I$ is open in $X$ and contains all continuous points of $f$. In particular $[a,b] = I \cup D$. For $k \geq 1$ define $D_k$ by $$D_k = \left(d_k-\frac{\delta}{4M\cdot 2^{k}}, \, d_k+\frac{\delta}{4M\cdot 2^{k}} \right) \cap [a,b].$$  Then $I$ and $D_i$ will cover $[a,b]$, and because $[a,b]$ is compact it is already covered by a finite union $$[a,b] = I \cup D_1 \cup D_2 \cup \dots \cup D_n$$ for some $n \geq 1$.  The complement $$Y = [a,b] \setminus (D_1 \cup \dots \cup D_n)$$ is compact (union of closed intervals) and contained in $I$. Therefore it can be covered by open intervals, such that on the closure of each interval, supremum $-$ infimum is less than $\delta/(b-a)$. By definition of $I$, every point has such a neighborhood.\\
	Also by compactness of $Y$, a finite subcover of such intervals exists. Now all end points of this subcover can be considered as points in the partition. The partition will divide $Y$ into finitely many closed intervals such that $\sup(f) - \inf(f) \leq \delta/(b-a)$ will hold on each interval.  Now we have the closure of $D_1 \cup \dots \cup D_n$ is itself a union of closed intervals with a total length less than $\delta/M$.
	
	The entire interval $[a,b]$ is now partitioned so that $U(f, P) - L(f, P) < 2\delta$. $\delta$ can be chosen arbitrarily small and $f$ is Riemann Integrable.	
	
	\item Finite sets are countable. The result follows directly from (2).
\end{enumerate}

\item $f, g$ should be bounded.
\begin{itemize}
	\item $0\leq \sup\{\abs{f(x)}: a \leq x\leq b\} = M < \infty$. For given $\epsilon>0$, $\exists P = \{a = x_0 < x_1 < \cdots < x_n=b\}$ s.t.
	$$\sum_{i=1}^n(x_{i}-x_{i-1})(M_i-m_i) < \frac{\epsilon}{2M+1}$$
	Since $$\abs{f(x)^2 - f(y)^2} \leq \abs{f(x)-f(y)}(\abs{f(x)} + \abs{f(y)}) \leq 2M\abs{f(x)-f(y)}$$
	, let $\tilde{M_i}, \tilde{m_i}$ be supremum and infimum of $f^2$ in $[x_{i-1}, x_i]$. Then $$\tilde{M_i} - \tilde{m_i} \leq 2M(M_i-m_i)$$
	Thus $$ \sum_{i=1}^n(x_{i}-x_{i-1})(\tilde{M_i} - \tilde{m_i}) \leq \sum_{i=1}^n (x_{i} - x_{i-1}) 2M(M_i -m_i) \leq 2M \cdot \frac{\epsilon}{2M+1} < \epsilon $$
	and $f^2$ is integrable.
	\item Now write $fg = \frac{1}{4}[(f+g)^2-(f-g)^2]$ to observe that $fg$ is integrable because $f+g, f-g, (f+g)^2, (f-g)^2$ are integrable.
\end{itemize}
\end{enumerate}
\end{document}