%!TEX encoding = utf-8
\documentclass[11pt]{report}
\usepackage{kotex}
\usepackage{amsmath}
\usepackage{amsfonts}
\usepackage{amssymb}
\usepackage{mathtools}
\usepackage{geometry}
\geometry{
	top = 20mm,
	left = 20mm,
	right = 20mm,
	bottom = 20mm
}
\geometry{a4paper}

\pagenumbering{gobble}
\renewcommand{\baselinestretch}{1.3}
\newcommand{\mf}[1]{\mathfrak{#1}}
\newcommand{\mc}[1]{\mathcal{#1}}
\newcommand{\bb}[1]{\mathbb{#1}}
\newcommand{\rmbf}[1]{\mathrm{\mathbf{#1}}}
\newcommand{\inv}{^{-1}}
\newcommand{\norm}[1]{\left\lVert#1\right\rVert}
\newcommand{\paren}[1]{\left( #1 \right)}
\renewcommand{\span}[1]{\left\langle #1 \right\rangle}
\newcommand{\adj}{\text{*}}
\newcommand{\ra}{\rightarrow}
\newcommand{\Ra}{\Rightarrow}
\newcommand{\abs}[1]{\left|#1\right|}
\newcommand{\ds}{\displaystyle}
\newcommand{\inte}{\mathrm{int}\,}
\newcommand{\imp}{\implies}
\newcommand{\bs}{\backslash}
\newcommand{\diam}{\text{diam}}
\newcommand{\dist}{\text{dist}}
\newcommand{\mimp}{$\implies$}
\newcommand{\miff}{$\iff$}
\newcommand{\R}{\mathbb{R}}
\newcommand{\N}{\mathbb{N}}
\newcommand{\Z}{\mathbb{Z}}
\newcommand{\Q}{\mathbb{Q}}
\newcommand{\C}{\mathbb{C}}

\usepackage{enumitem}
\setlist[enumerate,1]{label={\textbf{\sffamily\arabic*.}}}
\setlist[enumerate,2]{label={\textbf{\sffamily(\arabic*)}}}

\begin{document}
\begin{center}
\textbf{\Large 해석개론 및 연습 1 과제 \#4}\\
\large 2017-18570 컴퓨터공학부 이성찬
\end{center}
\begin{enumerate}
\item It is trivial that $ x $ is continuous on $\R$, thus finite dimensional polynomials are also continuous on $\R$, since polynomials are composed of sums and products of $x$.
\begin{enumerate}
	\item $(x+1)^3$ is continuous on $(-1, 1)$, and never 0 in this interval. Thus $f(x) = 1/(x+1)^3$ is continuous on $(-1, 1)$.\\
	Consider $x_n = 1/n - 1$. We immediately observe that $-1 < 1/n - 1 \leq 0$, and $\lim_{n\ra\infty}x_n = -1$. Thus $\span{x_n}$ is a Cauchy sequence in $(-1, 1)$. But $f(x_n) = n^3$, and the sequence diverges. Therefore $f(x)$ is not uniformly continuous on $X$. 
	\item $x+3$ is continuous on $(0, \infty)$ and never 0 in this interval. Thus $f(x) = 1/(x+3)$ is continuous on $ (0, \infty) $.\\
	For given $\epsilon > 0$, Set $\delta = 9\epsilon$. For $x, y\in (0, \infty)$, we observe that $$9 < xy + 3x + 3y + 9 = (x+3)(y+3)$$ then if $\abs{x - y} < \delta$, $$\abs{f(x) -f(y)} = \abs{\frac{1}{x+3} - \frac{1}{y+3}} = \frac{\abs{x - y}}{(x+3)(y+3)} < \frac{9\epsilon}{9} = \epsilon$$Thus $f(x)$ is uniformly continuous on $X$.
	\item $x^2+1$ is continuous on $\R$, and never 0. Thus $f(x) = 1/(x^2+1)$ is continuous on $\R$.\\
	Given $\epsilon >0$, set $\delta = \epsilon$. Since $$x^2y^2 +x^2+y^2+1 - x - y = x^2y^2 + \left(x - \frac{1}{2}\right)^2 + \left(y - \frac{1}{2}\right)^2+ \frac{1}{2} > 0$$
	, the following directly follows. $$\frac{x+y}{(x^2+1)(y^2+1)} < 1$$
	Now, if $\abs{x - y} < \delta = \epsilon$, $$\abs{f(x) - f(y)} = \abs{\frac{1}{x^2+1} - \frac{1}{y^2+1}} = \frac{\abs{x + y}\abs{x - y}}{(x^2+ 1)(y^2+1)} < \abs{x - y}  < \epsilon$$
	Therefore $f(x)$ is uniformly continuous on $X$.
	\item $x^2+1$ is continuous on $(0, \infty)$, and $\sqrt{x}$ is continuous on $(1, \infty)$. Thus their composition, $f(x) = \sqrt{x^2+1}$ is continuous on $(0, \infty)$.\\
	It is trivial that $$\sqrt{x^2+1} + \sqrt{y^2+1} - x - y > 0 \imp \frac{x+y}{\sqrt{x^2+1}+\sqrt{y^2+1}} < 1$$ so given $\forall \epsilon > 0$, if $\abs{x - y} < \delta = \epsilon$, we have $$\abs{f(x) - f(y)} = \abs{\sqrt{x^2+1} - \sqrt{y^2+1}} = \frac{(x+y)\abs{x - y}}{\sqrt{x^2+1}+\sqrt{y^2+1}} <\abs{x - y} < \epsilon$$
	Therefore $f(x)$ is uniformly continuous on on $X$.
\end{enumerate}

\item 
\begin{enumerate}
	\item (\mimp) Suppose $x_0 \in X'\bs X$. Then there exists a sequence $x_n$ in $X$ that converges to $x_0$. Because $f$ is uniformly continuous and $x_n$ is a Cauchy sequence, $f(x_n)$ is also a Cauchy sequence. Thus $\lim_{n\ra\infty} f(x_n) = \alpha \in \R$ exists.  Define the continuous extension $g$ of $f$ by setting $g(x_0) = \alpha$.\\
	Now we must check if $g(x_0)$ is well-defined. For any two sequence $\span{x_n}, \span{y_n}$ that converge to $x_0$, consider $\span{z_n} = x_1, y_1, x_2, y_2, \dots$. It is trivial that $z_n\ra x_0$. Since $\span{z_n}$ is a Cauchy sequence, $\span{f(z_n)}$ is also a Cauchy sequence by uniform continuity of $f$. Let its limit be $\gamma$. Then $\span{f(x_n)}, \span{f(y_n)}$ is a subsequence of $\span{f(z_n)}$, thus they both must converge to $\gamma$.\\
	Now we must check if $g$ is continuous on $\overline{X}$. For $x_0\in \overline{X}$, there exists a sequence $x_n$ in $X$ that converges to $x_0$. Since $g(x_n) = f(x_n)$, $f(x_n)$ converges to $f(x_0)$ by continuity of $f$. Thus $g(x)$ is continuous extension of $f$ to $\overline{X}$.\\
	\\
	($\impliedby$) Since $X$ is bounded, there exists a closed ball $B$ such that $X\subset B$. Because $\overline{X}$ is the smallest closed set containing $X$, $\overline{X} \subset B$, and $\overline{X}$ is bounded. We know that $\overline{X}$ is closed, thus $\overline{X}$ is compact. By Heine's Theorem, the continuous extension $g$ of $f$ is uniformly continuous on $\overline{X}$. Now $f$ is uniformly continuous since it is defined on a subset of the domain of $g$.
	\item Since $\overline{X} = \{(x, y)\in \R^2: x^2+y^2\leq 3 \}$, define the continuous extension $g$ of $f$ by $$g(x, y) = \begin{cases}
		f(x, y) & (x^2+y^2 < 3)\\
		\sqrt{x^{2020} + y^{2020} + x^2 +1} & (x^2 + y^2 = 3, x = \sqrt{3}\cos\theta, y = \sqrt{3}\sin\theta)
	\end{cases}$$
	(Such $\theta \in [0, 2\pi)$ exists) Now we show that $g(x, y)$ is continuous on $\overline{X}$. For $(x_0, y_0)\in X'\bs X$, $x_0^2 + y_0^2 = 3$, set $x_0 = \sqrt{3}\cos\theta, y_0 = \sqrt{3}\sin\theta$. Define a sequence in $X$ by $$(x_n, y_n) = \left(\left(\sqrt{3} - \frac{1}{n}\right)\cos\theta, \left(\sqrt{3} - \frac{1}{n}\right)\sin\theta \right)$$
	($x_n^2 + y_n^2 < 3$ can be easily checked), and it converges to $(x_0, y_0)$. It can be easily seen that $g(x_n, y_n) \ra g(x_0, y_0)$, because $1/n \ra 0$. Thus $g(x)$ is a continuous extension of $f$ to $\overline{X}$ and therefore $f$ is uniformly continuous on $X$.
	\item (\mimp) Since $f$ is uniformly continuous on $(a, b)$, there exists a continuous extension $g$ of $f$ to $[a, b]$. Since $g(x)$ is continuous at $x = a$, $\forall \epsilon > 0$, $\exists\,\delta$ s.t. $x\in \overline{X}$, $\abs{x - a} < \delta \imp \abs{g(x) - g(a)} <\epsilon$. Observe that $x\in \overline{X}, \abs{x-a} < \delta$ is equivalent to $x\in [a, a+\delta)$. Thus $(a, a + \delta) \subset [a, b]$, and we have $$x\in (a, a+\delta) \imp x\in [a, a+ \delta) \imp \abs{g(x) - g(a)} = \abs{f(x) - g(a)} < \epsilon$$Now by definition, $\lim_{x \ra a^+}f(x) = g(a)$.\\
	Similarly, since $g(x)$ is continuous on $x= b$, $\forall \epsilon > 0$, $\exists\,\delta$ s.t. $x\in \overline{X}$, $\abs{x - b} < \delta \imp \abs{g(x) - g(b)} <\epsilon$. Observe that $x\in \overline{X}, \abs{x-b} < \delta$ is equivalent to $x\in (b-\delta, b]$. Thus $(b-\delta, b) \subset [a, b]$, and we have $$x\in (b-\delta, b) \imp x\in (b - \delta, b] \imp \abs{g(x) - g(b)} = \abs{f(x) - g(b)} < \epsilon$$Now by definition, $\lim_{x \ra b^-}f(x) = g(b)$.
\end{enumerate}

\item Let the domain be $X = \R\bs\{0\}$.
\begin{enumerate}
	\item $\ds\lim_{x\ra0^-}\frac{\max\{x, 0\}}{x} = 0$.\\
	$\forall \epsilon> 0$, set $\delta = \epsilon$, $(-\delta, 0)\subset X$, and since $\max\{x, 0\} =0$ for all $x$ in this interval, $\frac{\max\{x, 0\}}{x} = 0 < \epsilon$.\\
	$\ds\lim_{x\ra0^+}\frac{\max\{x, 0\}}{x} = 1$.\\
	$\forall \epsilon> 0$, set $\delta = \epsilon$, $(0, \delta)\subset X$, and since $\max\{x, 0\} =x$ for all $x$ in this interval, $\abs{\frac{\max\{x, 0\}}{x}-1} = 0 < \epsilon$.\\
	Thus the wanted limit is $-1$.
	\item Given $\epsilon >0$, set $\delta = \sqrt{\epsilon}$. $(0, \delta) \subset X$, and if $0<x<\delta$, $0 < x^2 < \delta^2 = \epsilon$, then $0 < x^3/\abs{x} < \epsilon$. Since $x >0$, $\abs{x^3/\abs{x} - 0} < \epsilon$. $\therefore\ds \lim_{x\ra0^+}$$\frac{x^3}{\abs{x}} = 0$.
\end{enumerate}

\item 
\begin{enumerate}
	\item \textbf{True}. Since $f, g$ are uniformly continuous on $X$, $\forall \epsilon > 0$, $\exists\,\delta$ s.t. if $\norm{x - y} <\delta$ for all $x, y\in X$ $\imp \norm{f(x)-f(y)} < \epsilon/2$ and $\norm{g(x) - g(y)} < \epsilon/2$. We immediately have
	$$\norm{f(x) + g(x) - f(y) - g(y)} \leq \norm{f(x) - f(y)} + \norm{g(x) - g(y)} < \frac{\epsilon}{2}+ \frac{\epsilon}{2} < \epsilon$$ Thus $f+g$ is uniformly continuous on $X$. 
	\item \textbf{False}. (Counterexample) $f(x) = g(x) = x$ defined on $\R$. $f, g$ are uniformly continuous, but $x^2$ is not uniformly continuous (proof in textbook).
	\item \textbf{True}. We use the fact $\max\{f, g\} = \frac{f+g}{2} + \frac{\abs{f-g}}{2}$. For a constant $c$, since $cf$ is uniformly continuous if $f$ is uniformly continuous, it is sufficient to show that $\abs{f-g}$ is uniformly continuous, then the result directly follows by (1). Furthermore, because $f-g$ and $\norm{x}$ are uniformly continuous, ($\forall \epsilon > 0$, set $\delta = \epsilon$. Then $\norm{x - y} < \delta \imp \norm{\norm{x} - \norm{y}} \leq \norm{x - y} <\epsilon$) we show that their composition is uniformly continuous.\\
	\\
	\textbf{Claim}. Suppose $f:X\ra Y$, $g: Y\ra Z$ are uniformly continuous. Then $g\circ f$ is uniformly continuous.\\
	\textbf{Proof}. For given $\epsilon > 0$, $\exists\,\delta_1$ s.t. $\norm{x' - y'} < \delta_1 \imp \norm{g(x') - g(y')} < \epsilon$, for all $x', y'\in Y$. For this $\delta_1$, $\exists\,\delta$ s.t. $\norm{x-y} < \delta \imp \norm{f(x)-f(y)} < \delta_1$, for all $x, y\in X$. Then if $\norm{x - y} < \delta \imp \norm{f(x) - f(y)} < \delta_1 \imp \norm{g(f(x)) - g(f(y))} < \epsilon$. Thus $g\circ f$ is uniformly continuous.
\end{enumerate}

\item Define $g(x) = x^{2016} - f(x)$ on $[0, 1]$. Then since $x^{2016}, f(x)$ are continuous, its difference $g(x)$ is also continuous. Thus we have $$g(0) = 0 - f(0) \leq 0 \leq 1-f(1) = g(1)$$and by IVT, there exists $x_0\in [0, 1]$ s.t. $g(x_0) = 0$. For this $x_0$, $f(x_0) = x_0^{2016}$.

\item 
\begin{enumerate}
	\item Suppose $f$ is Hölder continuous. Given $\forall \epsilon > 0$, set $\delta = \left(\frac{\epsilon}{M}\right)^{1/\alpha}$. Since $y = x^\alpha$ $(\alpha > 0)$ is increasing, for all $x, y\in X$, if $\norm{x - y} < \delta$, $M\norm{x - y}^\alpha < \epsilon$. By the Hölder continuity condition, $\norm{f(x) - f(y)} < M\norm{x - y}^\alpha <\epsilon$, and $f$ is uniformly continuous on $X$.
	\item For fixed $x, y\in X$, suppose $x < y$. Given $N\in \N$, define $x_0, \dots, x_N$ as follows. $$x_0 = x,\; x_1 = x_0 +1\cdot \frac{y-x}{N},\; \dots, \;x_i = x_0 + i\cdot \frac{y-x}{N},\; \dots,\; x_N = y$$
	Then we have
	$$\norm{f(x) - f(y)} \leq \sum_{i=0}^{N-1}\norm{f(x_i) - f(x_{i+1})} \leq \sum_{i=0}^{N-1} M\norm{x_i - x_{i+1}}^\alpha = M  \frac{\norm{x- y}^\alpha}{N^{\alpha-1}}$$
	As $N\ra \infty$, $0 \leq \norm{f(x)- f(y)} \ds\leq \lim_{N\ra\infty}  M  \frac{\norm{x- y}^\alpha}{N^{\alpha-1}} = 0$. Thus $\norm{f(x) - f(y)} = 0$ for all $x, y\in X$. Thus $f(x) = f(0)$, merely a constant function.
\end{enumerate}

\item 
\begin{enumerate}
	\item Let $y\in f(\overline{A})$. Then there exists $x_0\in \overline{A}$ s.t. $f(x_0) = y$. Given $\epsilon > 0$, since $x_0\in \overline{A}$, there exists $\delta > 0$ s.t. $N_{\R^m}(x_0, \delta)\cap A \neq \emptyset$. Take an element $x$ from $N_{\R^m}(x_0, \delta) \cap A$. Since $x\in A $, $f(x) \in f(A)$, and since $x\in N_{\R^m}(x_0, \delta) $, $ f(x) \in N_{\R^n}(f(x_0), \epsilon)$ by continuity of $f$. Therefore we have $f(x) \in N_{\R^n}(f(x_0), \epsilon) \cap f(A) \neq \emptyset$, and $f(x_0) = y\in \overline{f(A)}$. $f(\overline{A}) \subset \overline{f(A)}$.
	\item False. Consider $f(x) = \frac{1}{x^2+1}$. $f$ is continuous. Set $A = (0, \infty)$. Then $\overline{A} = [0, \infty)$, $f(\overline{A}) = (0, 1)$, while $\overline{f(A)} = [0, 1]$.
\end{enumerate}

\item Define $f_A(x) = \dist(\{x\}, A)$, $f_B(x) = \dist(\{x\}, B)$. Then $$f(x) = \frac{f_B(x)}{f_A(x) + f_B(x)}$$
is a function that satisfies the requirements. The following should be checked.
\begin{enumerate}
	\item[(i)] Is $f: \R^d \ra [0, 1]$ ?\\
	The value of the $\dist$ function is always greater than equal to 0, and $f(x) \leq 1$ is trivial.\\
	Also note that the denominator is never $0$ since $A, B$ are disjoint.
	\item[(ii)] $x\in A \imp f(x) = 1$ ?\\
	If $x\in A$, $f_A(x) = 0$, $f_B(x) > 0$. ($A\cap B = \emptyset$) Thus $f(x) = \frac{f_B(x)}{0+f_B(x)} = 1$.
	\item[(iii)] $x\in B \imp f(x) = 0$ ?\\
	If $x\in B$, $f_B(x) = 0$, $f_A(x) > 0$. ($A\cap B = \emptyset$) Thus $f(x) = \frac{0}{f_A(x)+0} = 0$.
	\item[(iv)] Are $f_A, f_B$ continuous ?\\
	We will show that $f_A$ is uniformly continuous. For $x, y\in \R^d$, and any $a\in A$, the following holds by triangle inequality.
	$$\dist(\{x\}, \{a\}) \leq \dist(\{x\}, \{y\}) + \dist(\{y\}, \{a\})$$
	Taking infimum over all $a\in A$ gives
	$$\dist(\{x\}, A) \leq \dist(\{x\}, \{y\}) + \dist(\{y\}, A)$$
	Switching roles for $x, y$ will give us another inequality, and combining it with the above inequality will give us
	$$\abs{f_A(x)-f_A(y)} = \abs{\dist(\{x\}, A) - \dist(\{x\}, A)} \leq \dist(\{x\}, \{y\}) = \norm{x - y}$$
	Therefore, for all $\epsilon > 0$, setting $\delta = \epsilon$ will make $f_A$ satisfy the definition of uniform continuity. Thus $f_A$ is (uniformly) continuous.	The proof is symmetric for $f_B$.
	\item[(v)] Is $f$ continuous ?\\
	Since the denominator is never 0 and $f_A, f_B$ are continuous, $f$ is continuous.
\end{enumerate}

\item (\mimp) Define a function $F:X\ra X\times \R$ as $F(x) = (x, f(x))$. Then $F(X) = E$. To show compactness of $E$, we will show that $F$ is continuous.\\
For $x\in X$, consider a sequence $\span{x_n}$ in $X$ converging to $x$. ($X$ is closed) By the continuity of $f$, $\span{f(x_n)}$ converges to $f(x)$. Therefore $\span{F(x_n)}$ converges to $(x, f(x)) = F(x)$. Thus $F$ is continuous, and because $X$ is compact, its image $E=F(X)$ is also compact.\\
($\impliedby$) Since $E$ is compact, it is closed, and consider a sequence $(x_n, f(x_n))$ in $E$ that converges to $(x, f(x))$. Then $x_n$ is a sequence in $X$, and if $x_n \ra x$, $x\in X$. ($X$ is closed) Then we know that $f(x_n)$ must converge to $f(x)$ ($*$), which implies continuity of $f$ on $X$.
\\
($*$) If $(x_n, f(x_n))\ra (x, f(x))$ and $x_n\ra x$, then $f(x_n)\ra f(x)$.

\item We first prove the following inequality.\\
\textbf{Claim}. For convex function $f:(a, b)\ra \R$ and $a < x < y < z$, $$\frac{f(y) - f(x)}{y - x} \leq \frac{f(z) - f(y)}{z - y}$$
\textbf{Proof}. Since $f$ is convex, for $t\in (0, 1)$, $$f(tx +(1-t)z) \leq tf(x) + (1-t)f(z)$$, and set $y = tx + (1-t)z$. Multiply $z-x$ on both sides, then $$(z-x)f(y) \leq t(z-x)f(x) + (1-t)(z-x)f(z) = (z-y)f(x) + (y-x)f(z)$$
Rearranging the terms gives $$(z-y)(f(y)-f(x))\leq (y-x)(f(z)-f(y))$$
, which directly gives the inequality.\\
\\
Suppose $x\in (a, b)$. Since $(a, b)$ is open, select real numbers s.t. $x_0 < x_1 < x, y < x_2 < x_3$, and define $C = \max\{\frac{\abs{f(x_1) - f(x_0)}}{x_1-x_0}, \frac{\abs{f(x_3) - f(x_2)}}{x_3-x_2} \}$. Given $\epsilon > 0$, choose $\delta = \min\{\epsilon/C, x_2 - x_1\}$. 
If $x > y$, by \textbf{Claim}, $$\frac{f(x) - f(y)}{x-y} \leq \frac{f(x_2) - f(x)}{x_2-x} \leq \frac{f(x_3) - f(x_2)}{x_3-x_2}$$
and
$$\frac{f(x) - f(y)}{x -y} \geq \frac{f(y) - f(x_1)}{y - x_1} \geq \frac{f(x_1) - f(x_0)}{x_1-x_0} $$, therefore $$\abs{\frac{f(x)-f(y)}{x - y}} \leq C = \max\left\{\frac{\abs{f(x_1) - f(x_0)}}{x_1-x_0}, \frac{\abs{f(x_3) - f(x_2)}}{x_3-x_2} \right\}$$
and the inequality above can be shown similarly for $x < y$.\\
Therefore if $\abs{x - y} <\delta$, $\abs{f(x) - f(y)} \leq C\abs{x - y} \leq C \dfrac{\epsilon}{C} =\epsilon$. $f(x)$ is continuous.

\item Since $f$ is continuous, $f$ has a maximum at $\alpha_x \in [a, x]$, by EVT. Then $f^*(x) = f(\alpha_x)$.\\
If $x_1 < x_2$, $f^*(x_2) = \sup \{f(y):y\in[a, x_2] \} = \max\{f(a_{x_1}), \sup\{f(y):y\in[x_1, x_2]\} \} \geq f(\alpha_{x_1}) = f^*(x_1)$. Therefore $f^*$ is increasing.\\
For continuity, let $x_0\in X = [a, b]$, and let $f^*(x_0) = M (=\sup\{f(y): y\in[a, x_0] \})$.
\begin{enumerate}
	\item[Case 1.]  $f(x_0) < M$\\
	Since $f$ is continuous, for $\epsilon = M - f(x_0)$, there exists $\delta_1 > 0$ s.t. $\abs{x - x_0} < \delta_1, x\in X \imp \abs{f(x) - f(x_0)} < \epsilon$. Then $f(x) - f(x_0) < \epsilon \imp f(x) < M$, in this interval. Thus $\abs{x - x_0} < \delta_1, x\in X \imp f^*(x) = M$.
	\item[Case 2.] $f(x_0) = M$\\
	For all $\epsilon > 0$, $\exists\,\delta_2 > 0$ s.t. $\abs{x - x_0} < \delta_2, x\in X \imp \abs{f(x) - f(x_0)} < \epsilon \imp M - \epsilon < f(x) < M + \epsilon$. Therefore we have $M - \epsilon < f^*(x) < M + \epsilon$, by the continuity of $f$. Now we have $\abs{f^*(x) - f^*(x_0)} < \epsilon$. 
\end{enumerate}
For both cases, setting $\delta = \min\{\delta_1, \delta_2\}$ will give us $\abs{f^*(x) - f^*(x_0)} < \epsilon$. Thus $f^*$ is continuous.

\item Consider $$f(x) = \frac{1}{x}\sin \frac{1}{x} \quad x\in (0, 1]$$
Since $1/x$ and $\sin x$ are continuous on $(0, 1]$, $\R$ respectively, $f$ is also continuous on $(0, 1]$. Suppose $f$ attains maximum value $M$ at $x_0$. It is trivial that $M > 0$, because $M \geq f(2/\pi) = \pi/2 > 0$. From this result we also know that $\sin\frac{1}{x_0} > 0$.\\
Then for $x' = \left(1/x_0+ 2\pi \right)\inv (<1)$, $f(x') = \left(\frac{1}{x_0} + 2\pi\right)\sin\left(\frac{1}{x_0}+2\pi\right) = M + 2\pi \sin \frac{1}{x_0} > M$, contradicting the choice of $M$. Therefore $f$ has no maximum. Similarly, if $f$ attains minimum value $m$ at $x_1$, we know that $\sin \frac{1}{x_1} < 0$, and setting $x' = (1/x_1 + 2\pi)\inv$ will let us arrive at a contradiction, which contradicts the choice of $m$. Thus $f$ is continuous on $(0, 1]$ but has no minimum or maximum.



\end{enumerate}
\end{document}