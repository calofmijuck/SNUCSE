\section*{March 4th, 2019}
$A:B:C = 3+\epsilon : 4 +\epsilon : 3-2\epsilon$\\\\
\textbf{해석학}:
다항/지수/로그/초월함수 $\ra$ 미분 가능 함수 $\ra$ 연속 함수 $\ra$ 적분 가능 함수
(점점 더 \textit{나쁜} 함수를 배운다 - For application and curiosity)\\\\
\textbf{Overview}
\begin{center}
	\begin{tabular}[c]{|p{7cm}|p{7cm}|}
		\hline
		\textbf{해석개론 1} & \textbf{해석개론 2} \\ \hline
		$\R^d$ and its topology & 함수열 $\{f_n\}$ \\ \hline
		연속 함수 & Function Space \\ \hline
		미분 가능성 & Fourier Series \\ \hline
		Riemann-Siteltjes Integral & Lebesgue Integral \\\hline
	\end{tabular}
\end{center}
실수 $\R$
\begin{enumerate}
	\item Algebraic Structure (Field)
	\item \textbf{Ordered} Field
	\item 해석학적 구조, $\R$ vs $\Q$ ?
	\item Denseness: Ordered field $F$, $a, b\in F$, if $a<b$, $\exists\, r$ s.t. $a<r<b$
\end{enumerate}~\\
\textbf{Completeness of $\R$}
\begin{itemize}
	\item Bounded above
	\item Upper bound
	\item Least upper bound, supremum
\end{itemize}~\\
(Completeness) $\varnothing \neq S\subseteq R$, if $S$ is bounded above, $\sup S$ exists.\\
$\iff$ Monotonic Sequence Theorem\\
$\iff$ Cauchy sequence converges
\pagebreak