%!TEX encoding = utf-8
\documentclass[12pt]{report}
\usepackage{kotex}
\usepackage{amsmath}
\usepackage{amsfonts}
\usepackage{amssymb}
\usepackage{mathtools}
\usepackage{geometry}
\geometry{
	top = 20mm,
	left = 20mm,
	right = 20mm,
	bottom = 20mm
}
\geometry{a4paper}

\pagenumbering{gobble}
\renewcommand{\baselinestretch}{1.3}
\newcommand{\numl}[1]{\item[\large\textbf{\sffamily #1.}]}
\newcommand{\num}[1]{\item[\textbf{\sffamily #1}]}
\newcommand{\mf}[1]{\mathfrak{#1}}
\newcommand{\mc}[1]{\mathcal{#1}}
\newcommand{\bb}[1]{\mathbb{#1}}
\newcommand{\rmbf}[1]{\mathrm{\mathbf{#1}}}
\newcommand{\inv}{^{-1}}
\newcommand{\norm}[1]{\left\lVert#1\right\rVert}
\newcommand{\paren}[1]{\left( #1 \right)}
\renewcommand{\span}[1]{\left\langle #1 \right\rangle}
\newcommand{\adj}{\text{*}}
\newcommand{\ra}{\rightarrow}
\newcommand{\abs}[1]{\left|#1\right|}
\newcommand{\ds}{\displaystyle}
\newcommand{\inte}{\mathrm{int}}
\newcommand{\imp}{\implies}
\newcommand{\bs}{\backslash}
\newcommand{\diam}{\text{diam}}

\begin{document}
March 29th, 2019\\
\textbf{Remark}. $\limsup$ is the limit of $\sup$. If $\sup$ is easy to calculate, find $\sup$ and take the limit.\\\\
\textbf{Quiz 1 Solutions}\\
\#1. Given set $A$, $\inte(A)$, $A'$, determine whether the set is open or closed.
\begin{enumerate}
	\item $A = \bb{N}\subset \bb{R}$. $\inte(A) = \emptyset$, $A' = \emptyset$, $A$ is closed.
	\item $\bb{Q} \subset \bb{R}$. $\inte(\bb{Q}) = \emptyset$, $\bb{Q}' = \bb{R}$, $\bb{Q}$ is neither open nor closed.
	\item $C = [0, 1]\cup (2, 3)\cap \{4\}\subset \bb{R}$. $\inte(C) = (0, 1)\cup (2, 3)$, $C' = [0, 1]\cup [2, 3]$, $C$ is neither open nor closed.
	\item $D = \bigcup_{n=1}^\infty \{(\frac{1}{n}, y) : 0\leq y\leq 1 \}\subset \bb{R}^2$. $\inte(D) = \emptyset$, $D'=D\cup \{(0, y) :0\leq y\leq 1 \}$, $D$ is neither open nor closed. ($\because \inte D \neq D$, $\overline{D}\neq D$)
\end{enumerate}~\\
\#2. Find a limit point of given set.
\begin{enumerate}
	\item $A = \bb{Q}\subset\bb{R}$. $0$ is a limit point. (Directly follows from Archimedes' principle)
	\item $B = \{\frac{1}{n}: n\in \bb{N} \}\subset\bb{R}$. $0$ is a limit point of $B$. (Also directly follows from Archimedes')
	\item $C = \{2^{-n} + 3^{-m}: n, m\in \bb{N} \} \subset \bb{R}$. $0$ is a limit point of $C$. Given $\epsilon > 0$, exists $N\in \bb{N}$ such that for $n, m\geq N$, $2^{-n} < \epsilon/2$, $3^{-m} < \epsilon / 2$. Then $0\neq 2^{-n} + 3^{-m} < \epsilon$.
\end{enumerate}~\\
\#3. True or False? If false, find a counterexample.
\begin{enumerate}
	\item $\overline{A\cup B} = \overline{A} \cup \overline{B}$ \textbf{True}
	\item $ \overline{A\cap B} = \overline{A} \cap \overline{B} $ \textbf{False}. Set $A = (0, 1), B = (1, 2)$. \\\textbf{Correct Statement}: $ \overline{A\cap B} \subset \overline{A} \cap \overline{B}$
	\item $ \inte(A\cup B) = \inte(A) \cup \inte(B) $ \textbf{False}. Set $A = [0, 1], B = [1, 2]$. \\\textbf{Correct Statement}: $\inte(A) \cup \inte(B) \subset \inte(A\cup B)$
	\item $ \inte(A\cap B) = \inte(A) \cap \inte(B) $ \textbf{True}
\end{enumerate}
\pagebreak
\textbf{Thm}. $A\subset B \implies \overline{A}\subset \overline{B}, \inte(A)\subset \inte(B)$.\\
\textbf{Proof}. 
\begin{itemize}
	\item We need to show $A'\subset B'$. Let $x\in A'$.\\
	$\implies$$\forall \epsilon > 0$, $N(x, \epsilon)\cap (A\backslash \{x\}) \neq\emptyset$. \\
	$\implies \forall \epsilon > 0, N(x, \epsilon)\cap (B\backslash \{x\})\neq \emptyset$\\
	$\implies x\in B'$.
	\item Let $x\in \inte(A)$\\
	$\implies \exists \epsilon > 0, N(x, \epsilon) \subset A \implies N(x, \epsilon) \subset B \implies x\in \inte(B)$.
\end{itemize}
~\\
\textbf{Proof of (c).} $A, B\subset A\cup B$ \\$\implies \inte(A), \inte(B)\subset \inte(A\cup B)$. Thus $\inte(A)\cup \inte(B)\subset \inte(A\cup B)$\\
\\
\textbf{Proof of (d).} $A\cap B \subset A, B \implies \inte(A, B) \subset \inte(A), \inte(B)$. Thus $\inte(A\cap B)\subset \inte(A)\cap \inte(B)$
Suppose $x\in \inte(A)\cap \inte(B)$. Then $\exists \epsilon_A, \epsilon_B > 0$ s.t. $N(x, \epsilon_A)\subset A, N(x, \epsilon_B)\subset B$. Take $\epsilon = \min\{\epsilon_A, \epsilon_B \}/ 2$. Then $N(x, \epsilon) \subset A, B$. Therefore $N(x, \epsilon) \subset A\cap B$, $x\in \inte(A\cap B)$.
~\\
\textbf{Example.} $A = \{(x, y):x^2+2y^2<1 \}$. $\inte(A) = A, A' = \{(x, y): x^2+2y^2\leq 1 \}$.\\
Suppose $(x_0, y_0) \in A$. $x_0^2 + 2y_0^2 = 1-\delta < 1$ for some $\delta >0$. By symmetry, let $x_0, y_0 >0$. From
$$(x_0+\epsilon)^2 + 2(y_0+\epsilon)^2 = x_0^2 +2y_0^2 + \epsilon(2x_0 + 4y_0 + 3\epsilon) < 1$$
, we want $\epsilon(2x_0+4y_0+3\epsilon) < \delta$. Set $\epsilon < 1/10$. Then $\epsilon(2x_0 + 4y_0 + 3\epsilon) < \epsilon(2x_0 + 4y_0 +3) < \delta$. Now set $\epsilon = \min\left\{\ds \frac{1}{2 (2x_0+4y_0+3)}, \frac{1}{100}\right\} > 0$.\\ Then $\abs{x - x_0}<\epsilon, \abs{y-y_0}<\epsilon$. $x_0^2+2y_0^2 < (x_0+\epsilon) ^2 + 2(y_0+\epsilon)^2 <1$. $N((x_0, y_0), \epsilon) \subset A$.\\
Interior points are limit points, and for the points $(x_0, y_0)$ on the border, consider a sequence $(x_0-1/n, y_0-1/n)$. Then the elements are in $A$ and they converge to $(x_0, y_0)$. Thus the border is also included in $A'$.
\pagebreak

April 1st, 2019\\
$ \inte A : x\in A$ s.t. $N(x, \epsilon) \subset A$ for some $\epsilon > 0$.\\
$A': x\in \bb{R}^d$ s.t. $N(x, \epsilon) \cap (A\backslash\{x\})\neq \emptyset$ for $\forall \epsilon >0$\\
$\overline{A}: x\in \bb{R}^d$ s.t. $N(x, \epsilon) \cap A\neq\emptyset$, $\forall \epsilon>0$, $\overline{A} = A\cup A'$\\
\textbf{Example}. $A=[0, 1)\cup \{2\}$. $ 1\in A' $, $ 2\notin A' $, $ 2\in \overline{A} $\\
\textbf{Prop 2.3.3} $ x\in A' \implies N(x, \epsilon)\cap (A\backslash \{x\})$ 는 무한집합이다.\\
\textbf{Proof}. 유한집합이라고 가정하자. $N(x, \epsilon) \cap (A\backslash \{x\}) = \{x_1, \dots, x_n\}$ 이라 할 수 있다.
Set $\delta = \min\{\norm{x-x_i}: \forall i\}$. Then $ N(x, \delta) \cap (A\backslash \{x\})=\emptyset $. 모순.\\
그래서 사실은 공집합이 아닌 것으로 정의했지만 \textbf{사실은} 무한집합이다.\\
\textbf{Remark}. $A'\neq\emptyset\implies A$는 무한집합.\\
(대우) $A$가 유한집합이면 극한점이 존재하지 않는다. (2.2 보기 4)\\
(역) \textbf{거짓}. $A = \{1, 2, \cdots\}$ 이면 $A'=\emptyset$.\\
그러면 역이 언제 성립하나요? 다음 단원 내용!\\
\\
\textbf{Definition}. Convergence in $\bb{R}^d$\\
Let $\span{x_n}$ be a sequence in $\bb{R}^d$. $$\lim_{n\rightarrow \infty} x_n= x \iff \forall \epsilon>0, \exists N \text{ s.t. } \left (n\geq N \implies \norm{x_n-x}<\epsilon\right )$$\\
\\
\textbf{Exercise}. $x_n = (x_n^{(1)}, \dots)$, $x = (x^{(1)}, \dots)$ 일 때, $x_n\rightarrow x \iff \forall i, x_n^{(i)} \rightarrow x^{(i)} $\\
\\
\textbf{Notation}. $A\subset \bb{R}^d$; $\span{x_n}$ is a sequence in $A$ $\iff \forall n, x_n\in A $\\
\\
\textbf{Theorem 2.2.2}.
\begin{enumerate}
	\item $x\in A' \iff \exists \span{x_n}$ in $A\backslash \{x\}$ such that $x_n\rightarrow x$
	\item $x\in \overline{A} \iff \exists \span{x_n}$ in $A$ such that $x_n\rightarrow x$ 
\end{enumerate}
\textbf{Proof}.
\begin{enumerate}
	\item ($\imp$) $x_n\in N\left(x, \frac{1}{n} \right) \cap (A\backslash \{x\})$ 이라 하자. (공집합이 아니므로 이러한 원소가 존재한다.) 그러면 $\norm{x_n-x}<1/n$ 이므로 $x_n$ 은 $x$ 로 수렴한다. 그리고 $x_n\in A\backslash\{x\}$ 이므로 수열이 $A\backslash\{x\}$ 에 있다.
	\item Left as exercise. Replace $A\bs \{x\} $ with $A$.
\end{enumerate}~\\
\\
\textbf{Theorem 2.2.3}. The following are equivalent.
\begin{enumerate}
	\item $F$ is closed.
	\item $F'\subset F$.
	\item $F =\overline{F}$
	\item For a sequence $\span{x_n}$ in $F$, $\ds\lim_{n\rightarrow \infty} x_n = x$ $\imp$ $x\in F$.
\end{enumerate}
\textbf{Proof}. \\
(1)$\iff$(3) ($\overline{F}$: smallest closed set containing $F$.)\\
(2)$\iff$(3) 은 자명.\\
(1)$\iff$(4) by the above theorem. (Thm 2.2.2)\\
\\
\textbf{Applications}.
\begin{enumerate}
	\item $A'$ is closed.\\
	\textit{Proof}. We want to show that $(A')' \subset A'$.\\
	We want to show: $x\in (A')' \imp x\in A'$.\\
	($A'$ 이 공집합이면 자명. 공집합이 아니라고 가정하고...)\\
	Given $\epsilon>0$, $N(x, \epsilon)\cap (A'\bs \{x\}) \neq \emptyset$. Take an element $y\in A'$ from this set. Now set $\delta = \min\{\norm{x-y}, \epsilon - \norm{x-y} \}$ then we have $N(y, \delta) \cap (A\bs \{y\}) \neq \emptyset$. ($\because y\in A'$)\\
	$z \in N(y, \delta) \cap (A\bs \{y\})$ 라 하자.
	\begin{enumerate}
		\item $z\in A\bs \{y\} \subset A$.
		\item $\norm{x-z} \leq \norm{x-y} + \norm{y-z} < \norm{x-y} +\delta \leq \epsilon$ ($z\in N(y, \delta)$)
		\item $\norm{x-z} \geq \norm{x-y}-\norm{y-z}>\norm{x-y}-\delta \geq 0$ (By the choice of $\delta$.) Thus $x\neq z$.
	\end{enumerate}
	Therefore $z\in N(x, \epsilon)$ (by (b)), $z\in A\bs \{x\}$ (by (a), (c)).\\
	$x\in A'$ since $N(x, \epsilon)\cap (A\bs \{x\})$ is not empty.
	\item $A\subset \bb{R}$: closed and bounded $\imp$ $\inf A = \min A$, $\sup A = \max A$. (Existence)\\
	\textit{Proof}. Let $\sup A=x\notin A$. ($\sup A\in A$ 이면 자명)\\
	\textit{Claim}. $x\in A'$.\\
	\textit{Proof of Claim}. $\forall \epsilon>0$, $N(x, \epsilon) = (x-\epsilon, x+\epsilon)$\\
	$x = \sup A$ 이므로 $x-\epsilon$ is not an upper bound.\\
	$\exists y$ such that $y \in (x-\epsilon, x)$\\
	$y\in N(x, \epsilon) \cap (A\bs \{x\})\neq \emptyset$ 이므로 $x$ 는 극한점.\\
	따라서 $x\in A' \subset A$ (closed set 이므로 Thm 2.2.3 (2)) 모순. \\
	$\sup A\in A$ 이므로 이 값이 최댓값이다. 
\end{enumerate}
\pagebreak
\textbf{2.3 유계집합과 코시수열}\\
핵심: Thm 2.3.4, Thm 2.3.7\\
\textbf{Definition}. $\span{x_n}$: 유계수열(bounded sequence) $\iff$ $\exists M>0$ s.t. $\norm{x_n}\leq M$ for all $n\in \bb{N}$.\\
\\
\textbf{Definition}. $n_1<n_2<\cdots$ : sequence in $\bb{N}$ 이라 하자. $\span{x_{n_k}}_{k=1}^\infty = (x_{n_1}, x_{n_2}, \dots)$ 를 $\span{x_n}$의 부분수열(subsequence)이라 한다. \\
\\
\textbf{Theorem 2.3.4} (Bolzano-Weierstrass Theorem)\\
If $\span{x_n}$ is bounded, there exists a convergent subsequence of $\span{x_n}$.\\
\\
\textbf{Idea of Proof}. Equivalent formulation for sets.\\
\\
\textbf{Definition}. Set $A$ is bounded $\iff$ $\exists M>0$ such that $\norm{x}<M$ for all $x\in A$.\\
\\
\textbf{Theorem 2.3.2} (Equivalent of 2.3.4) $A$가 유계이고 무한집합이면, $A'\neq \emptyset$.\\
\\
\textbf{Remark}. $A'\neq \emptyset \imp A$: 무한집합.\\
역이 성립하기 위해서는 $A$가 유계라는 조건이 필요하다.\\
\\
극한점이 중요한 이유는 계속 수열과 관련이 있기 때문이다.\\
\textbf{Example}. $A = \{1/n: n\in \bb{N}\}$ 을 고려하는 것은 수열 $x_n=1/n$ 을 고려하는 것이나 마찬가지이다. 이 수열 $x_n$ 이 $x$ 로 수렴하는 것은 $A'=\{x\}$ 와 동치이다. (Hence the name ``limit point")\\
이로부터 $x\in A' \iff $ Exists a subsequence of $\span{x_n}$ in $A\bs \{x\}$ converging to $x$.\\
\\
\textbf{Proof of 2.3.2}
\begin{enumerate}
	\item \textbf{Lemma 2.3.1} 축소구간정리 in $\bb{R}^d$.\\
	$B$ is a closed box in $\bb{R}^d$ $\iff$ $B = I_1\times I_2 \times \cdots \times I_d$, where $I_i = [a_i, b_i]$ for $i = 1, \dots, d$. ($I_i$ is a closed and bounded interval.)\\
	$$B_1\supset B_2\supset \cdots \imp \bigcap_{n=1}^\infty B_n \neq \emptyset$$\\
	\textbf{Proof}. 각 `좌표' $I_i$ 별로 1차원 축소구간정리를 적용하면 된다.
	\item \textbf{Divide and Conquer Strategy}\\
	B: Box 일 때, $\diam(B) = \sup\{\norm{x-y}: x, y\in B \} = \sqrt{(a_1-b_1)^2 + \cdots + (a_d-b_d)^2}$\\
	\textbf{Claim}. There exists closed boxes $B_1, B_2, \dots$ s.t. 
	\begin{enumerate}
		\item $B_1\supset B_2\supset \cdots$
		\item $\diam B_n = \dfrac{1}{2^n}\diam B_1$
		\item $B_n\cap A$: 무한집합
	\end{enumerate}
	\textbf{Proof}. (Induction) $n = 1$; $B_1$: 충분히 커서 $A\subset B_1$ 인 box 를 잡으면 된다.\\
	Suppose we have $B_1, \dots, B_n$; $B_n$을 $2^d$ 등분하면 적어도 하나는 $A$의 원소를 무한개 포함하고 있다. 그 집합을 $B_{n+1}$ 으로 잡는다. (비둘기집의 원리)\\
	이제 $x\in \bigcap_{n=1}^\infty B_n$ 으로 잡으면 (축소구간정리에 의해 잡을 수 있다) $x\in A'$. ($A'\neq \emptyset$)
	$\because \forall \epsilon>0$, $\diam B_n <\epsilon$ 인 $N\in \bb{N}$ 을 찾아 $n\geq N$ 일 때 부등식이 성립하도록 할 수 있다. 이러한 $n$ 들에 대하여 $B_n\subset N(x, \epsilon)$. 그러면 $N(x, \epsilon)\cap (A\bs \{x\}) \supset B_n\cap (A\bs \{x\})$.
\end{enumerate}
\pagebreak
April 3rd, 2019\\
우리가 지금 2.3 을 하고 있는데, 2 가지 중요한 결과가 있어요.\\
\\
\textbf{Theorem 2.3.4} $\span{x_n}$ 이 bounded 이면 수렴하는 부분수열을 갖는다.\footnote{증명이 가장 테크니컬 해요!}\\
\\
\textbf{Theorem 2.3.2} $A$가 유계인 집합이고 무한집합이면 극한점을 가진다. $A'\neq \emptyset$ \\
증명은 축소구간정리를 박스로 확장해가지고 분할 정복하면 된다.\\
\\
\textbf{Recall 2.3.3} $x \in A' \imp$ $N(x, \epsilon)\cap (A\bs \{x\}) $ 는 무한집합이다.\\
\\
\textbf{Proof of 2.3.4}. $ A = \{x_1, x_2, \dots, x_n\} $ 라고 하면 이 집합은 유계이다. (수열이 유계이므로)
\begin{enumerate}
	\item $A$가 유한집합: 자명.\\
	$ \exists x$ such that $x$ appears infinitely many times in $\span{x_n}$. (PHP) 이 경우에는 부분수열을 $x, x, \dots$ 로 잡으면 된다. 이는 수렴하는 부분수열이다.
	\item $ A $가 무한집합\footnote{이제 Thm 2.3.2 를 사용할 수 있다. 사실 경우를 나눈 것은 예외적인 case 를 처리하기 위한 것이었다.}\\
	$A'\neq \emptyset$ 이므로 $\alpha\in A'$ 이라 하자.\\
	\textbf{Claim}. $\exists n_1<n_2<\dots$ such that $\norm{x_{n_k}-\alpha} < 1/k$.\\
	\textbf{Proof}. (첨자들이 증가하면서 가까워져야 한다는 것이 유일하게 tricky 한 부분이다. 귀납법을 사용하자.) $k=1$: $x_{n_1} \in N(\alpha, 1)\cap (A\bs \{\alpha\})$ 로 잡으면 된다.\\
	$x_{n_1}, \cdots, x_{n_k}$를 잡았다고 가정: $N(\alpha, \frac{1}{k+1})\cap (A\bs \{\alpha\})$ 에서 $x_{n_{k+1}}$를 잡아야 하는데 이 집합은 무한집합이다. (Recall 2.3.3) 이 집합에서 첨자가 $n_k$보다 큰 항이 반드시 존재하므로 그 중 하나를 $x_{n_{k+1}}$ 이라 잡으면 된다.\\
	따라서 $\ds \lim_{k\rightarrow \infty} x_{n_k} = \alpha$ (Check as exercise)
\end{enumerate}~\\
\textbf{Application}. (Characterization of $\limsup$ and $\liminf$)\\
$x_n$ 이 bounded 이면, $A = \{x:\exists \text{ subsequence of } x_n \text{ converging to } x \}$. 이 때 Theorem 2.3.4에 의해 $A \neq \emptyset$ 임을 증명하였다.
\begin{enumerate}
	\item $A$: closed and bounded $\imp$ $\max(A), \min(A)$ 가 존재한다.\\
	\textbf{Proof}. $B= \{x_1, x_2, \dots \}$, $C = \{ \span{x_n} \text{에 무한 번 나타나는 수} \}$ 로 잡자. $A = B'\cup C, C\subset B, C'\subset B'$ 임을 확인해보라! 이를 이용하면 $B'\cup C = (B'\cup C')\cup C = B'\cup (C'\cup C) = B'\cup \overline{C}$ 가 되어 닫힌집합의 합집합은 닫힌 집합이다. $A$는 closed and bounded 이다.
	\item $\limsup x_n = \max(A)$, $\liminf x_n=\min(A)$\\
	(부분수열이 가질 수 있는 극한값들 중 가장 큰 값이 $ \limsup $, 가장 작은 값이 $ \liminf $)\\
	\textbf{Proof}. Recall\\
	$$ \limsup x_n = \alpha \iff \begin{cases}
		\text{(i) } \forall\epsilon > 0, \exists N \text{ s.t } (n\geq N\imp x_n<\alpha+\epsilon) \\
		\text{(ii) } \forall \epsilon>0, x_n>\alpha-\epsilon \text{ for infinitely many }n
	\end{cases} $$
	\begin{enumerate}
		\item 부분수열 $\span{x_{n_k}}\ra \beta$ 이면 (i)에 의해 $k\geq N \imp x_{n_k}<\alpha+\epsilon$ 이 되어 $\beta \leq \alpha + \epsilon$. $\beta \leq \alpha$. 그러므로 $\max(A) \leq \alpha$ 이다.
		\item $\forall \epsilon>0$, (i), (ii)에 의해 $x_n\in (\alpha-\epsilon, \alpha+\epsilon)$ 인 $n$ 이 무한히 많다. 이 유계인 구간에 속하는 수열의 항들에 대해 부분수열을 잡아 (further subsequence) $\gamma$ 로 수렴하도록 할 수 있다. (Theorem 2.3.4) 그러면 $\span{x_{m_k}} \ra \gamma \in [\alpha-\epsilon, \alpha+\epsilon]$. 따라서 $\alpha-\epsilon\leq \gamma\leq \max(A)$ 가 되어 $\alpha \leq \max(A)$.
	\end{enumerate}
	따라서 $\max(A) = \alpha$.
\end{enumerate}~\\
\\
\textbf{Definition}. $\span{x_n}$: \textbf{Cauchy Sequence} $\iff$ $\forall\epsilon>0$, $\exists N$ s.t. $[m, n\geq N \imp \norm{x_m-x_n}<\epsilon]$\\
\\
\textbf{Prop 2.3.6, Thm 2.3.8} $\span{x_n}$: convergent $\iff$ $\span{x_n}$: Cauchy sequence\footnote{중간고사 전 까지 가장 중요한 정리.}\\
\textbf{Proof}. ($\imp$) 자명. $\norm{x_m-x_n} \leq \norm{x_m-\alpha} + \norm{x_n-\alpha} < \epsilon/2+\epsilon/2 = \epsilon$ 인 $m, n\geq N$ 존재.\\
($\impliedby$) 수렴 값이 없는 상태에서 증명해야 한다. 먼저 수렴 값을 찾아보자.
\begin{enumerate}
	\item $\span{x_n}$ is bounded.\\
	\textbf{Proof}. $\exists N$ s.t. $\norm{x_m-x_n} < 1$ for all $m, n\geq N$.\\
	Set $M = \max\{\norm{x_1}, \dots, \norm{x_{N-1}}, \norm{x_N}+1 \}$. ($\norm{x_m} < \norm{x_N}+1$)\\
	따라서 $\norm{x_n}\leq M$ for all $n\in\bb{N}$.
	\item There exists a subsequence $\span{x_{n_k}}$ converging to some $\alpha$. (Thm 2.3.4)
	\item $\span{x_n}$ converges to $\alpha$.\\
	\textbf{Proof}. $\epsilon>0$ 에 대해, 
	\begin{enumerate}
		\item 코시 수열의 성질에 의해 $\exists N_1$ s.t. $\norm{x_m-x_n}<\epsilon/2$ for all $m, n\geq N_1$.
		\item 부분수열이 $\alpha$로 수렴하므로 $\exists N_2$ s.t. $\norm{x_{n_k} -\alpha} < \epsilon/2$ for all $k\geq N_2$. 
	\end{enumerate}
	Let $N = \max\{N_1, N_2 \}$. $ n\geq N_1, n_N\geq n_{N_1} \geq N_1 $ 이므로,\\
	$$n>N \imp \norm{x_n - \alpha}\leq \norm{x_n-x_{n_N}} + \norm{x_{n_N} - \alpha} < \frac{\epsilon}{2} + \frac{\epsilon}{2} < \epsilon$$
\end{enumerate}
\pagebreak
\textbf{Remark}. 우리의 여정을 돌아보자. 
\begin{enumerate}
	\item Archimedes' Principle 을 가정하면\\
	Completeness Axiom $\imp$ Monotone Convergence Theorem $\imp$ 축소구간정리 $\imp$ Bolzano-Weierstrass Theorem $\imp$ \textbf{Cauchy Convergent Theorem}\footnote{In any metric spaces, this is the condition for completeness.} \\
	(Exercise) $\imp$ Completeness Axiom
	\item \textbf{Example}. $X = C([0, 1])$. (Set of functions that are continuous in $[0, 1]$) How would we define $\norm{f-g}$? $\int_0^1 \abs{f(x)-g(x)}dx$ ? $\max\{\abs{f(x)-g(x)}: x\in[0, 1]\}$ ? Only the second choice gives completeness for $X$.
	\item \textbf{Convergence Test} without limit value. (\textbf{Theorem 2.3.9}) \\
	$\sum_{n=1}^{\infty} a_n$ is convergent $\iff$ $\forall\epsilon > 0$, $\exists N$ s.t. ($n > m \geq N \imp \abs{a_{m+1} + \cdots + a_n} < \epsilon$)\\
	\textbf{Proof}. Trivial.
\end{enumerate}~\\
\textbf{Definition}. $\sum a_n$ is \textbf{absolutely convergent} $\iff$ $\sum \abs{a_n}$ is convergent\\
\\
\textbf{Theorem}. An absolutely convergent series converges.\\
\textbf{Proof}. Suppose $\sum \abs{a_n}$ converges. For $\forall\epsilon>0$, there exists $N$ such that $\abs{\abs{a_{m+1}}+\cdots + \abs{a_n}}<\epsilon$ for all $m, n\geq N$. Therefore, for $m, n\geq N$, $$\abs{a_{m+1}+\cdots + a_n} \leq \abs{a_{m+1}}+\cdots + \abs{a_n} < \epsilon$$
and $\sum a_n$ converges.






\end{document}