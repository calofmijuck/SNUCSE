\documentclass[12pt]{article}
\usepackage{geometry}
\usepackage{xcolor}
\usepackage{listings}

\geometry{
	top = 20mm,
	left = 25mm,
	right = 25mm,
	bottom = 20mm
}
\geometry{a4paper}
%\renewcommand{\familydefault}{\sfdefault}
\renewcommand{\baselinestretch}{1.3}

\definecolor{mGreen}{rgb}{0,0.6,0}
\definecolor{mGray}{rgb}{0.5,0.5,0.5}
\definecolor{mPurple}{rgb}{0.58,0,0.82}
\definecolor{mRed}{rgb}{1, 0, 0}
\definecolor{backgroundColour}{rgb}{0.95,0.95,0.92}

\lstdefinestyle{CStyle}{
	commentstyle=\color{mGreen},
	keywordstyle=\color{mPurple},
	numberstyle=\tiny\color{mGray},
	stringstyle=\color{mGreen},
	basicstyle=\ttfamily\footnotesize,
	emph = {randkth, randomized\_partition, srand, time, rand, swap, randomized\_select, deterministic\_select, detkth, checker, malloc, partition, median, printf},
	emphstyle=\color{red},
	breakatwhitespace=false,         
	breaklines=true,                 
	captionpos=b,                    
	keepspaces=true,                 
	numbers=left,                    
	numbersep=-10pt,                  
	showspaces=false,                
	showstringspaces=false,
	showtabs=false,                  
	tabsize=4,
	language=C++,
}

%opening
\title{\sffamily Algorithms: Selection in Linear Time}
\author{2017-18570 Sungchan Yi\\Dept. of Computer Science and Engineering}
\date{April 6th, 2019}
\begin{document} 

\maketitle
\tableofcontents

\section{Introduction}
In this homework, we implement the randomized and deterministic selection algorithms with C++ language. The deterministic selection algorithm must run in $O(n)$ time. Given $n$ elements in an array and an integer $k$ ($0\leq k < n$), the algorithm must find the $k$-th smallest element in the array.\footnote{For simplicity in programming, we will count from 0.} Additionally, we implement a checker program for the correctness for the implemented algorithms. The following are required:
\begin{itemize}
	\item Implement the randomized and deterministic selection algorithms.
	\item Run each algorithm for given inputs, measure the running time, print the $k$-th smallest element.
	\item Check the correctness of each algorithm by using a checker program that runs in $O(n)$ time. Print the result of checking.
	\item After measuring the running time for inputs of various sizes, compute the ratios of the constant hidden in the asymptotic complexities of each algorithm.
\end{itemize}
The following are the restrictions on given inputs.
\begin{itemize}
	\item $n, k$ are integers, with $1\leq n\leq 1,000,000,000$, $0\leq k < n$.
	\item All the elements given in the array are distinct.\footnote{This condition is required for the deterministic selection algorithm to run in $O(n)$ time.}
\end{itemize}

\section{Implementation}
In this section, we will cover the implementation of each algorithm. The basic idea is divide and conquer. Select a pivot and partition the given array into subproblems. The two algorithms discussed here differs only on the method of selecting the pivot.\\
For the codes, assume the necessary header files are already included.
\subsection{Randomized Select}
\begin{lstlisting}[style=Cstyle]
	int randomized_select(int a[], int n, int k) {
		return randkth(a, 0, n - 1, k);
	}
\end{lstlisting}
The \textsc{Randomized-Select} algorithm chooses a pivot randomly from the array. The algorithm is implemented in \texttt{randkth} method. \texttt{randkth(arr, st, ed, k)} will return the $k$-th smallest element from \texttt{arr[st..ed]}.\footnote{\texttt{A[i..j]} denotes the elements from \texttt{A[i]}, ..., \texttt{A[j]}.}
\begin{lstlisting}[style=Cstyle]
	int randkth(int arr[], int st, int ed, int k) {
		if(st == ed) // base case for recursion
			return arr[st];
		int pivot = randomized_partition(arr, st, ed); // index of pivot
		
		// index of pivot from strarting from st
		int idx = pivot - st + 1; 
		
		// Divide and Conquer
		if(k < idx) // search left
			return randkth(arr, st, pivot - 1, k);
		else if(k > idx) // search right
			return randkth(arr, pivot + 1, ed, k - idx);
		else // element is found
			return arr[pivot];
	}
\end{lstlisting}~\\
For \texttt{randomized\_partition}, select a random index from \texttt{st..ed} and use it as a pivot.
\begin{lstlisting}[style=Cstyle]
	int randomized_partition(int arr[], int st, int ed) {
		srand(time(NULL));
		int pIdx = st + rand() % (ed - st + 1); // random index
		swap(arr, ed, pIdx); // put it at the end of array
		int x = arr[ed]; // pivot element
		int i = st - 1; // track the correct place of pivot
		for(int j = st; j < ed; ++j) {
			if(arr[j] <= x)
				swap(arr, ++i, j); // increment i then swap
		}
		swap(arr, ++i, ed); // put pivot at correct place
		return i;
	}
\end{lstlisting}~\\
\texttt{swap(A, i, j)} will swap \texttt{A[i]} and \texttt{A[j]}. Its implementation is trivial, so we omit it here.\\
According to the CLRS textbook, the above algorithm will run in expected $O(n)$ time. But the worst case complexity is $O(n^2)$, since we get extremely unlucky by the random function choosing the smallest/largest element from \texttt{A[st..ed]}. That choice of pivot will lead to an unbalanced partition of input array.\\

\subsection{Deterministic Select}
\begin{lstlisting}[style=Cstyle]
	int deterministic_select(int a[], int n, int k) {
		return detkth(a, 0, n - 1, k);
	}
\end{lstlisting}
The deterministic selection algorithm chooses a pivot by using ``median of medians". First divide the array into $\lceil n/5\rceil$ sub-arrays, each with 5 elements, except for the last sub-array. Then choose a median from each sub-array and from the chosen medians, choose a median and use it as the pivot for partition. \texttt{detkth(arr, st, ed, k)} will return the $k$-th smallest element from \texttt{arr[st..ed]}.
\begin{lstlisting}[style=Cstyle]
	int detkth(int arr[], int st, int ed, int k) {
	    if(st == ed) // base case
	    	return arr[st];
		int n = ed - st + 1; // number of elements
		
		// array for storing medians of each sub-array
		int* med = (int*) malloc((n + 4) / 5 * sizeof(int));
		// find medians of each subarray
		int i;
		for(i = 0; i < n / 5; ++i)
			med[i] = median(arr + st + 5 * i, 5);
		if(5 * i < n) {
			med[i] = median(arr + st + 5 * i, n % 5);
			i++;
		}
		
		// choose median of medians
		int medOfMed = (i == 1) ? med[i-1] : detkth(med, 0, i-1, i/2);
		
		// partition the array with median of medians
		int pivot = partition(arr, st, ed, medOfMed);
		int idx = pivot - st + 1;
		
		// divide and conquer
		if(idx == k)
			return arr[pivot];
		else if(idx > k)
			return detkth(arr, st, pivot - 1, k);
		else
			return detkth(arr, pivot + 1, ed, k - idx);
	}
\end{lstlisting}~\\
\texttt{i == 1} in line 18 was to handle the case with the input array size less than 6. To find the median for each sub-array, \texttt{median(arr, n)} method was called, and it will find the median of $n$ elements starting from \texttt{arr} (pointer) by insertion sort.
\begin{lstlisting}[style=Cstyle]
	int median(int arr[], int n) {
		for(int i = 1; i < n; i++) {
			for(int j = i; j > 0; j--) {
				if(arr[j] < arr[j - 1]) swap(arr, j, j - 1);    
				else break;
			}
		}
		return arr[n / 2];
	}
\end{lstlisting}~\\
\texttt{partition} process is done similarly.
\begin{lstlisting}[style=Cstyle]
	int partition(int arr[], int st, int ed, int x) {
		// search for x in arr, and move it to the end
		int i;
		for(i = st; i < ed; ++i) {
			if(arr[i] == x) break;
		}
		swap(arr, i, ed);
		i = st - 1;
		for(int j = st; j < ed; ++j) {
			if(arr[j] <= x) // increment i then swap arr[i] and arr[j]
				swap(arr, ++i, j);	
		}
		swap(arr, ++i, ed);
		return i;
	}
\end{lstlisting}~\\
According to the textbook, the above algorithm will run in $O(n)$ time.\\

\subsection{Checker}
To check the correctness, we implement a checker program. The basic idea is to count the number of elements less than or equal to the return value of above algorithms.
\begin{lstlisting}[style=Cstyle]
	bool checker(int a[], int n, int k, int ans){
		int cnt = 0;
		for(int i = 0; i < n; ++i) {
			if(ans >= a[i]) cnt++; // count elements
		}
		if(cnt == k) // cnt must be k to be true
			return true;
		else return false;
	}
\end{lstlisting}~\\
This checker runs in $O(n)$ time because it sweeps the given array and compares each element with \texttt{ans}.

\section{Experiments}
Here is how the test was done.
\begin{itemize}
	\item Test environment
	\begin{itemize}
		\item OS: Ubuntu 18.04.2 LTS
		\item CPU: Intel® Core™ i5-6200U CPU @ 2.30GHz $\times$ 4
		\item RAM: 8GB  
	\end{itemize} 
	\item Checker Code\\
	\texttt{chrono} header file was used to check the wall clock.
	\begin{lstlisting}[style=Cstyle]
	#include <chrono>
	using namespace std::chrono;
	... // get input and etc. omitted here
	auto start = high_resolution_clock::now(); // start time
	int ans1 = randomized_select(arr, n, k);
	auto stop = high_resolution_clock::now(); // end time
	
	// check correctness
	if(checker(arr, n, k, ans1)) printf("%s","correct");
	else printf("%s","incorrect");
	
	auto duration = duration_cast<microseconds>(stop - start);
	printf(", Execution time ");
	std::cout << duration.count() / 1000.0 << " ms\n";
\end{lstlisting}
	Similar code was also run for \texttt{deterministic\_select} algorithm.
	\item The checker code was saved to \texttt{checker.cpp}.
	\begin{itemize}
		\item Compile: \texttt{g++ check.cpp -o check}
		\item Run: \texttt{./check.cpp < 1.txt}
	\end{itemize}
\end{itemize}
There were no reports of incorrectness during the following experiments.

\subsection{Randomized Select}
Since the running time of randomized algorithms differ for every execution, we ran the algorithm 10,000 times and take the average execution time.\\
\begin{center}
	\begin{tabular}{|c|c|c|c|c|}
		\hline
		Array Size ($n$) & 20,000 & 60,000 & 200,000 & 600,000 \\\hline
		Average Time (ms) & 0.363094 & 0.926259 & 1.955946 & 9.627423 \\\hline
	\end{tabular}\\\vspace{0.5em}
{\small Table 1: Average running time of \textsc{Randomized-Select} algorithm}
\end{center}
The estimated complexity is $T(n) = 0.01615n - 335.4$ ($\mu s$), with $R^2= 0.9789$. The $R^2$ value is less than the deterministic select due to the randomness of the algorithm.
\subsection{Deterministic Select}
The running time for deterministic algorithm does not differ very much. We ran it once here.
\begin{center}
	\begin{tabular}{|c|c|c|c|c|}
		\hline
		Array Size ($n$) & 20,000 & 60,000 & 200,000 & 600,000 \\\hline
		Time (ms) & 1.922 & 6.265 & 17.785 & 54.067 \\ \hline
	\end{tabular}\\\vspace{0.5em}
{\small Table 2: Running time of deterministic select algorithm}
\end{center}
The estimated complexity is $T(n) = 0.09034n + 269.3$ ($\mu s$), with $R^2 = 0.9996$.\\
Just for curiosity, we ran the program 1,000 times to see the average.

\begin{center}
	\begin{tabular}{|c|c|c|c|c|}
		\hline
		Array Size ($n$) & 20,000 & 60,000 & 200,000 & 600,000 \\\hline
		Average Time (ms) & 0.902151 & 2.726879 & 10.423119 & 31.067172 \\ \hline
	\end{tabular}\\\vspace{0.5em}
{\small Table 3: Average running time of deterministic select algorithm}
\end{center}
It can be easily seen that the deterministic algorithm show linearity. 


\section{Conclusion}
We calculated the ratio of running times by dividing the results of Table 2 by Table 1, for each $n$.
\begin{center}
	\begin{tabular}{|c|c|c|c|c|}
		\hline
		Array Size ($n$) & 20,000 & 60,000 & 200,000 & 600,000 \\\hline
		Ratio & 5.293 & 6.764 & 9.093 & 5.616 \\ \hline
	\end{tabular}\\\vspace{0.5em}
	{\small Table 4: Ratio of running times for each $n$}
\end{center}
With only the leading coefficients of the complexities for each algorithm, we can see that the ratio is $0.09034 / 0.01615 = 5.593808$. We conclude that the deterministic selection algorithm is around 5.5 times slower than the randomized selection algorithm.

\end{document}
