%!TEX encoding = utf-8
\documentclass[12pt]{report}
\usepackage{kotex}
\usepackage{amsmath}
\usepackage{amsfonts}
\usepackage{amssymb}
\usepackage{mathtools}
\usepackage{geometry}
\usepackage{graphicx}
\usepackage{subfig}
\geometry{
	top = 25mm,
	left = 25mm,
	right = 25mm,
	bottom = 25mm
}
\geometry{a4paper}

\pagenumbering{gobble}
\renewcommand{\baselinestretch}{1.5}
\newcommand{\numl}[1]{\item[\textbf{#1}]}
\newcommand{\num}[1]{\item[{#1}]}
\newcommand{\mf}[1]{\mathfrak{#1}}
\newcommand{\mc}[1]{\mathcal{#1}}
\newcommand{\bb}[1]{\mathbb{#1}}
\newcommand{\rmbf}[1]{\mathrm{\mathbf{#1}}}
\newcommand{\trans}{^{\mathrm{\mathbf{t}}}}
\newcommand{\inv}{^{-1}}
\newcommand{\norm}[1]{\left\lVert#1\right\rVert}
\newcommand{\paren}[1]{\left( #1 \right)}
\renewcommand{\span}[1]{\left\langle #1 \right\rangle}
\newcommand{\im}{\text{im}\:}
\newcommand{\rk}{\text{rk}}
\newcommand{\tr}{\text{tr}}
\newcommand{\diag}{\text{diag}}
\newcommand{\adj}{\text{*}}
\renewcommand{\inv}{^{-1}}
\newcommand{\tperp}{^\text{\sffamily perp}}
\newcommand{\nsub}{\mathrel{\unlhd}}
\newcommand{\pnsub}{\mathrel{\lhd}}

\begin{document}
\begin{center}
\textbf{\large 선형대수학 2} (2018. 10. 27.)
\end{center}

\begin{itemize}
\numl{1--4.} $F^n$ 에는 dot product 가 주어져 있다고 가정한다.
\numl{1.} (10점) $\bb{R}^2$의 두 벡터 $Y=\left(-3, \sqrt{3}\right)\trans$, $Z=(-3, -\sqrt{3})\trans$에 대하여,\\
(가) 두 reflection $S_Y, S_Z$ 에 $S_Y\circ S_Z$ 는 $\bb{R}^2$ 의 rotation 임을 보여라.\\
(나) $S_Y\circ S_Z = R_\theta$ 인 $\theta$ 를 구하라.

\numl{2.} (10점) (가) $W=\span{\left(-1, 1, 0\right)\trans, \left(1,0,-1\right)\trans} < \bb{R}^3$ 일 때, Gram-Schmidt Orthogonalization Process 를 이용하여 $W$ 의 orthonormal basis $\mf{C}=\{X_1, X_2\}$ 를 구하라.\\
(나) $\mf{C}$ 를 포함하는 $\bb{R}^3$ 의 orthonormal basis $\mf{B}=\{X_1, X_2, X_3\}$ 를 구하라. \\
(다) $Y=\left(2, 3, 4\right)\trans\in \bb{R}^3$ 와 가장 가까운 $W$ 의 vector 를 구하라.
 
\numl{3.} (10점) $S:\bb{R}^n\rightarrow\bb{R}^n$ 이 reflection 이면, $\det S = -1$ 임을 보여라.

\numl{4.} (10점) $\diag(1, -1, -1, -1)$ 은 $\bb{R}^4$ 의 reflection 인가? Rotation 인가?

\numl{5.} (10점) $A\in \mf{M}_{m, n}(F)$ 일 때, $\rk(A^\ast \cdot A) = \rk(A)$ 임을 보여라.

\numl{6.} (10점) $A, B\in \mc{M}_{2, 2}(F)$ 일 때, $\span{A, B} = \tr(A\trans\cdot \overline{B})$ 로 정의하자.\\
(가) $\span{ \:,\:}$ 는 $\mc{M}_{2, 2}(F)$ 의 inner product 임을 보여라. \\
(나) 위에서 정의한 inner product space $\big(\mc{M}_{2, 2}(F), \span{\:, \:}\big)$ 와 dot product 가 주어진 $F^4$ 가 inner product space 로서 isomorphic 함을 구체적인 isomorphism 을 통하여 증명하라.

\numl{7.} (10점) $\rmbf{V}_n(F) = \big\{(a_{ij})\in \rmbf{GL}_n(F) \:|\: a_{ij} = 0 \text{ if } i > j, a_{ii} = 1 \text{ for all } i \big\}$ 라고 표기할 때, $F^3$ 가 $\rmbf{V}_3(F)$ 와 isomorphic 한 group 이 되도록 $F^3$ 에 이항연산을 정의하라. (증명 필요 없음.)

\numl{8.} (10점) Cayley's Theorem 을 쓰고 증명하라.

\numl{9.} (10점) $N\unlhd G$ 일 때, quotient group $G/N$ 의 이항연산을 정의하고, 이 연산이 well-defined 되어 있음을 보여라.

\numl{10.} (10점) First Isomorphism Theorem 과 `학부 대수학의 반'을 이용하여, cyclic group 을 (모두) 분류하라.

\end{itemize}

\end{document}