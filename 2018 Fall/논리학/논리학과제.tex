%!TEX encoding = utf-8
\documentclass[12pt]{report}
\usepackage{kotex}
\usepackage{amsmath}
\usepackage{amsfonts}
\usepackage{amssymb}
\usepackage{mathtools}
\usepackage{geometry}
\geometry{
	top = 20mm,
	left = 20mm,
	right = 20mm,
	bottom = 20mm
}
\geometry{a4paper}

\pagenumbering{gobble}
\renewcommand{\baselinestretch}{1.3}

\begin{document}
\begin{center}
	~\\~\\~\\~\\~\\~\\~\\
\textbf{\Huge 논리학 중간 과제}\\~\\~\\~\\~\\~\\
\textbf{\large 문제 제공자} \\~\\
\begin{tabular}{rl}
	\textbf{학과/이름 :}& 컴퓨터공학부 이성찬\\
	\textbf{전화 번호 :}& 010-4030-6402\\
	\textbf{이메일 :}& calofmijuck@snu.ac.kr\\
\end{tabular}
~\\~\\~\\~\\~\\~\\
\textbf{\large 문제 풀이자} \\~\\
\begin{tabular}{rl}
	\textbf{학과/이름 :}& \underline{\qquad\qquad\qquad\qquad\qquad}\\
	\textbf{전화 번호 :}& \underline{\qquad\qquad\qquad\qquad\qquad}\\
	\textbf{이메일 :}& \underline{\qquad\qquad\qquad\qquad\qquad}\\
\end{tabular}

\end{center}
\pagebreak

\section*{1. 예문 1} 
세 명의 교수가 식당에 앉아 있습니다. 웨이터가 와서 다음과 같이 말했습니다.
\begin{center}
	``세 분 \textbf{모두} 커피를 주문하시는군요."
\end{center}
그랬더니 첫 번째 교수는 다음과 같이 답했습니다.
\begin{center} 
	``잘 모르겠습니다."
\end{center}
두 번째 교수는 다음과 같이 답했습니다.
\begin{center} 
	``잘 모르겠습니다."
\end{center}
마지막으로, 세 번째 교수가 말했습니다.
\begin{center} 
	``아니요, 모두가 커피를 원하지는 않습니다."
\end{center}
웨이터는 주문을 받고 돌아와 커피를 주문한 교수님들께 커피를 드렸습니다. \footnote{출처: http://math.zxcvber.com/problems/2018.html, 생성 날짜 2018/10/10, 방문 날짜 2018/11/10.} \\
\pagebreak

\section*{2. 논증 1}
웨이터의 질문 ``세 분 \textbf{모두} 커피를 주문하시는군요." 에 대하여, 다음과 같은 논증을 생각하자.\\~\\
$p$: 첫 번째 교수가 커피를 주문했다. \\
$q$: 두 번째 교수가 커피를 주문했다. \\
$s$: 잘 모르겠다고 대답했다.\\~\\
첫 번째 교수가 커피를 주문하지 않았다면, 잘 모르겠다고 대답하지 않았을 것이다.($\sim p \supset \sim s$)\\
두 번째 교수가 커피를 주문하지 않았다면, 잘 모르겠다고 대답하지 않았을 것이다.($\sim q \supset \sim s$)\\
---------------------------------------------------------\\
두 번째 교수는 커피를 주문하지 않았다.($\sim q$)\\~\\
위 논증이 타당한지 \textbf{완전 진리표 방법}을 이용해 조사하고, 커피를 주문한 교수님들이 누구인지 조사하시오.
\pagebreak

\section*{3. 논증 2}
위 논증 1 에서 얻은 결론을 바탕으로 다음 논증을 생각해 보자.\\~\\
$p$: 말을 끝까지 들었다. \\
$q_1$: 첫 번째 교수님이 웨이터의 질문에 대한 답을 알고 있다.\\
$q_2$: 두 번째 교수님이 웨이터의 질문에 대한 답을 알고 있다.\\
$q_3$: 세 번째 교수님이 웨이터의 질문에 대한 답을 알고 있다.\\~\\
첫 번째 교수님은 말을 끝까지 듣지 않아 모르겠다고 대답했다. ($\sim p \supset \sim q_1$)\\
두 번째 교수님은 말을 끝까지 듣지 않아 모르겠다고 대답했다. ($\sim p \supset \sim q_2$)\\
세 번째 교수님은 말을 끝까지 들었기에 모르겠다고 대답하지 않았다. ($\sim p \supset q_3$)\\
---------------------------------------------------------\\
말을 끝까지 들어봐야 한다. ($p$)\\~\\
이 논증은 타당한가?
\pagebreak

\section*{4. 예문 2}
옛날 한 마을에 어머니와 오누이가 살았어요.\\하루는 어머니가 산 너머 잔칫집에 일하러 가야 됐어요.\\
``호랑이가 노리고 있을지도 모르니 절대로 문을 열어 주면 안 된다. 알겠지?"\\
어머니는 하루 종일 잔칫일을 거들고 서둘러 산을 넘어 집으로 향했지요.\\그런데 고개 하나를 넘자 호랑이가 나타났어요.\\
``\textbf{떡 하나 주면 안 잡아먹지.}"\\
그래서 어머니는 떡을 주었고, 고개를 무사히 넘어갔어요.\\
그런데 고개가 여러 개 였는데, 고개를 넘을 때마다 호랑이가 나타나지 뭐예요.\\
결국 호랑이는 떡을 모조리 먹어치우고 어머니까지 꿀꺽 하고 삼켜버렸어요.\\
(후략) \footnote{해와 달이 된 오누이 - 출처: https://blog.naver.com/wjdgywls48/220661758083, 생성 날짜 2016/3/22, 방문 날짜 2018/11/10.}\\
\pagebreak

\section*{5. 논증 3}
다음과 같은 논증을 생각하자.\\~\\
$p$: 어머니가 고개를 만난다.\\
$q$: 어머니가 떡을 하나 잃어버린다.\\
$r$: 어머니가 떡이 없다.\\
$s$: 호랑이가 어머니를 잡아먹는다.\\~\\
어머니가 고개를 만나면 떡을 하나 잃어버린다. ($p \supset q$) \\
어머니가 고개를 만났는데 떡이 없으면 호랑이에게 잡아먹힌다. ($p\cdot r \supset s$)\\
호랑이가 어머니를 잡아먹었다. ($s$)\\
---------------------------------------------------------\\
어머니가 고개를 만났을 때 떡이 없었다. ($p\cdot r$)\\~\\
이 논증은 타당한가?
\end{document}