%!TEX encoding = utf-8
\documentclass[12pt]{report}
\usepackage{kotex}
\usepackage{amsmath}
\usepackage{amsfonts}
\usepackage{amssymb}
\usepackage{mathtools}
\usepackage{geometry}
\geometry{
	top = 20mm,
	left = 20mm,
	right = 20mm,
	bottom = 20mm
}
\geometry{a4paper}

\pagenumbering{gobble}
\renewcommand{\baselinestretch}{1.3}

\begin{document}
\begin{center}
	{\Large	\textbf{논리학 수시 과제}}\\
	2017-18570 컴퓨터공학부 이성찬
\end{center}

\textbf{예문 1 - ‘외계인 방문 가능성’을 언급한 NASA 문건의 진실} \footnote{\texttt{http://nownews.seoul.co.kr/news/newsView.php?id=20181208601005}, 2018.12.8 생성, 2018.12.8 방문}\\

미 항공우주국(NASA)의 한 과학자가 외계인이 이미 지구를 방문했을지도 모른다고 언급한 사실이 보도되어 일반의 비상한 관심을 끌고 있다.

지난 12월 3일(현지시간) 폭스 뉴스 (Fox News)는 NASA의 과학자가 외계인이 지구를 방문했을 가능성이 있다는 제목의 기사를 보도했다. (중략)
그러나 콜롬바노는 '라이브 사이언스'와의 인터뷰에서 폭스 뉴스와 그밖의 언론 보도는 자신의 진의를 왜곡하여 허위 사실을 보도한 것이라고 밝혔다. 
콜롬바노는 "그 문건은 지난봄 SETI (외계 지적 생명체 탐사) 연구소 회의에서 발표된 프리젠테이션이었다"고 해명하며 "이번 회의는 연구 프로그램의 미래 방향에 관한 과학자들의 피드백을 얻는 것이었다"고 밝혔다.
(중략) 다만 콜롬바노는 외계인이 지구를 방문했을 가능성이 있다는 열린 자세를 피력했을 뿐이다. 

SETI는 주로 생물학적 기원에 대한 증거로 우주에서 무선 신호를 스캔하여 외계 생명체를 찾기 위한 조직으로, UFO 보고서 및 기타 자료를 통해 체계적으로 수집하며, 우주에서 오는 외계 지성체의 희미한 신호를 찾아내기 위해 활동하고 있다.\\\\
\textbf{다음 논증의 타당성을 {\sffamily 벤 다이어그램} 또는 {\sffamily 오일러 다이어그램}으로 조사하시오.}\\
어떤 외계인은 지구를 방문했다.\\
지구를 방문하는 모든 것은 SETI가 관측했다.\\
---------------------------------------------------------------------\\
어떤 외계인은 SETI가 관측했다.\\\\\\\\\\\\\\\\\\\\\\\\\\\\



\pagebreak



\textbf{예문 2 - 애플, 또 한국 차별... 애플케어 + 가입 차단} \footnote{\texttt{http://www.zdnet.co.kr/news/news\_view.asp?artice\_id=20181207170337}, 2018.12.7 생성, 2018.12.8 방문}\\

애플이 지난 5일 오후부터 국내 구입한 아이폰·아이패드·애플워치 등 제품을 대상으로 무상보증 연장 프로그램인 애플케어플러스(AppleCare+) 가입 차단에 나섰다. 애플케어플러스에 이미 가입한 기기 대상으로 수리도 더 이상 제공하지 않을 방침이다.

애플케어는 애플이 판매하는 무상보증기간 연장 프로그램이다. 제품 구입 후 90일간으로 제한된 전화 기술지원 기간과 무상보증기간을 제품 구입 후 2년으로 연장해 준다(맥북프로·아이맥은 구입 후 3년). 단 제품 구입 후 미국은 60일 이내, 일본은 30일 이내 가입해야 한다.

미국이나 일본 등을 대상으로 판매되는 프로그램인 애플케어플러스는 전화 상담과 무상보증기간 연장 이외에 추가 혜택을 제공한다. 소비자의 과실로 인한 파손에 대해 최대 2회까지 상대적으로 저렴한 비용으로 수리나 제품 교환(리퍼)이 가능하다.

미국에서는 아이폰XS 화면이 파손되면 279달러(한국 35만 5천원), 이외의 손상에 대해서는 549달러(한국 69만 5천원)를 청구한다. 그러나 애플케어플러스에 가입한 경우 화면 손상은 29달러(약 3만 3천원), 이외 손상은 99달러(약 11만원)만 부담하면 된다.

한국은 현재 애플케어만 적용되는 국가다. 이처럼 수리 비용이 상승하면서 일부 소비자들은 국내 구매한 아이폰이나 아이패드, 애플워치에 미국이나 일본 등 해외 애플 온라인에서 애플케어플러스를 적용해 수리비 부담을 줄이기도 했다.

그러나 복수 온라인 커뮤니티에 따르면, 애플은 지난 5일 오후부터 국내에서 구입한 애플 제품에 대한 애플케어플러스 가입을 차단하고 있다. 신용카드 일련 번호 조회를 통해 미국이나 일본 등 해당 국가에서 발급된 카드가 아니면 가입을 거부하고 있다는 것이다.

뿐만 아니라 애플은 해외 애플케어플러스가 적용된 기기의 국내 수리도 일부 제한에 들어갈 것으로 보인다. 그 동안은 애플 가로수길 지니어스나 애플 공인 수리업체의 재량에 따라 이 기기들의 수리나 교환이 가능했다. (후략)
\\\pagebreak\\
\textbf{다음 조건적 삼단논법 논증의 타당성을 {\sffamily 진리수 방법}으로 조사하시오.}\\
애플케어+에 가입하려면 미국 신용카드를 발급받아야 하고 신용카드의 발급 비용은 \$250이다.\\
만약 내가 애플케어+를 가입하면 아이폰XS 화면 파손의 수리비로 총 \$279를 지출해야 한다.\footnote{수리비와 카드 발급 비용. \$29 + \$250}\\
만약 내가 애플케어+를 가입하지 않으면 아이폰XS 화면 파손의 수리비로 \$279를 지출해야 한다.\\
나는 미국 신용카드를 발급받거나, 애플케어+를 가입하지 않아야 한다.\\
---------------------------------------------------------------------\\
나는 아이폰XS 화면 파손의 수리비로 \$279를 지출해야 한다.\\


\end{document}